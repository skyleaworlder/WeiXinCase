\documentclass[UTF8]{ctexart}
\usepackage{graphicx}
\usepackage{subcaption}
\usepackage{dcolumn}
\usepackage{listings}
\usepackage{color}

\definecolor{codegreen}{rgb}{0,0.6,0}
\definecolor{codegray}{rgb}{0.5,0.5,0.5}
\definecolor{codepurple}{rgb}{0.58,0,0.82}
\definecolor{backcolour}{rgb}{0.95,0.95,0.92}

\lstset{
    backgroundcolor=\color{backcolour},
    commentstyle=\color{codegreen},
    keywordstyle=\color{magenta},
    numberstyle=\tiny\color{codegray},
    stringstyle=\color{codepurple},
    basicstyle=\footnotesize,
    breakatwhitespace=false,
    breaklines=true,
    captionpos=b,
    keepspaces=true,
    numbers=left,
    numbersep=5pt,
    showspaces=false,
    showstringspaces=false,
    showtabs=false,
    tabsize=2
}

\title{惟心馆血案}

\begin{document}

\maketitle

二十一世纪二十年代,黄渡大学济心学堂部分学生学习生活状况综述。

\section{序章}

2021 年春天,我在黄渡大学内过着普通人的生活,日子单调,却不算乏味。因为我在这并不短暂的一段时间里,有幸结识了一位名为张诺的少女。

这位同学是正常学龄入学,比我低一个年级、小两岁,思维清晰、头脑敏捷、认真仔细,最擅长的事情莫过于在一群人中脱颖而出。什么校会实验室打工、国创上创、三学期满绩,反倒是这些特质的附属品了。或许你会因身边缺少这样的时间管理大师而怀疑我的说法,在此我非常抱歉地声明:本人所言非虚,但无意传播焦虑。

我先前的室友就曾和我提起过张诺,三句两句就扯到了 “她如此强大的原因”。黄渡大学没有校园论坛,也没有由学生运营且受到大众认可的超人气信息源。好吧,或许曾经有过,也有过大规模的宣传,但都因为缺少推广、界面不美观、存在安全漏洞等问题无疾而终。到头来,大家仍只是使用各自擅长的通讯软件作为信息交换的工具。换句话说,校内信息十分闭塞。拥有高质量信息源的学生,能够毫不费力地攫取更大的好处。就读该校两年有余,我 OCIQ 上仅有 270 位本校好友,而张诺仅在 WeChat 就有 2100 个;同时,我所在的消息资源群数量也远不及她。这两点再加上她具备超强信息处理能力的大脑,恐怕就是其远超同龄人的直接缘由了。

~\\

今天是周六,我难得早起一趟,美好的半天从张诺约我一起吃教师食堂开始。两个菜,一碗饭,吃完就去图书馆九楼无意义内卷,晚上去大食堂一楼茄子鸡柳土豆随便选一样,十点半闭馆再骑车回寝,洗脸刷牙后上床穿越异世界……

“那个你听说了吗?” 她打断了我正在进行着的对今天的规划。

“什么东西。”

“就是昨天那件事。”

我不知道怎么回复她,继续夹菜吃饭。她不说是什么,我怎么可能知道。不过大概率是我认识的谁开始谈恋爱了,毕竟春天到了。

“啧,这个事情不方便直接说,我再提示下——‘惟心馆’,有印象吗?”

哦,惟心馆啊,我还真不熟。惟心馆是黄渡大学内的一座学院大楼,传媒与艺术学院所属,地面两层、地下一层,其地理位置较为偏僻。众所周知,黄渡大学由 “同勤”、“同美”、“济心”、“济德” 四座学堂组成,如果从这个角度谈起,惟心馆是 “同勤学堂” 其中一座学院楼。

我对那个地方十分陌生,尽管它距离我常去的图书馆仅有不到 100 米的距离,但我从未踏入那里一步。其外部的设计、内部的构造,我是一点也不了解。但既然张诺这么说,惟心馆肯定出了一件大新闻,并且我的社交圈内都在转发这件事。可惜我今天 11 点起床,看到张诺早上 8 点发的消息后就立刻奔向了食堂,连手机都还没看一眼。

但等我打开手机,翻了翻空间动态,却没有发现一条信息是与 “惟心馆” 三字相关。

我心中叹了口气。这也难怪,我才 270 个黄渡好友,其中大部分都是本专业的同学。惟心馆是张诺她们学堂的大楼,她能获得这个消息而我获取不到,也算是个很正常的事。

大概是猜到我没有找到结果的她,用手碰了碰我。

我抬头看去,张诺没有作声,也没有其他肢体动作,仅是用口型表示要说的话。我是万万不敢相信,因为她说的是:

“死人了。”

~\\

“惟心馆夜里发生离奇事件。今夜凌晨 1 时左右,黄渡大学的宁静被一声凄异怪叫划破。据昨晚惟新馆夜班保安所述,怪叫来自该馆阅览室。惟新馆地上两层,地下一层,内设大小两座影院,一间阅览室。因叫声极为短暂,两位保安未能迅速锁定声音方向。在检查二楼情况后,其中一位保安通过玻璃门到达天台,天台空无一人。最后,两位保安出馆巡查,在馆外发现草地上躺有一女子。”

“经鉴定,该女子于凌晨 1 时死亡,死因为胸骨断裂刺穿肺部导致的缺氧性窒息。”

张诺言罢,我们也来到了惟心馆。

\section{那块 U 盘}

我校由四个学堂组成,一般学生只能享受校内大约 1/4 的便利。这不是写在《学生管理办法》里的条例,而是大家心中的 “戒律” —— “每天都很忙,生活没有其他颜色,活着已经很累了,视线最好放在近处,不要去自己不该去的地方,不要听自己不该听的事情……” 这些 “戒律” 就像是刻在每个学生内心深处一般,每时每刻都发挥着它们的作用。

我们这些就读于黄渡大学的学生,都持有一张身份卡,这张卡则会标识持卡人的身份。

我是 “济心学堂” 的学生,张诺则是 “同勤学堂” 的。

由于 “戒律” 的存在,我甚至本不可能认识在 “同勤学堂” 生活的她。

但张诺是个例外。或者说,她与我这种 “大多数” 不同。她穿的是我没听过的衣服,哼的是我没听过的曲子,讲的是我没听过的故事。我起初并不清楚张诺她接触我们的缘由,甚至还把她当成骗子骂过。但随着对她的了解深入,我毫不怀疑她到哪里都会是那个消息最灵通的人,到哪里都会发光,她的光芒甚至会令周围的人感到刺眼。吃完饭后,我们沿着学校的主干道一路向西南骑行,就在我们前往惟心馆的路上,我跟她说我想看电影了,她就告诉我那里每周二、周五晚上都会开放观影;我给她吐槽图书馆环境不好,她就说顶楼的落地窗最配笔记本和咖啡。很明显,那应该不是她的生活,可我也无法想象。

~\\

“就这点?” 现在,我们站在惟心馆前。听了她对昨晚所发生之事的摘要后,我问道。

“就这点。”

“那你就相信了吗,你还是个玩 WeChat 的,朋友圈上面现在谣言那么多,这么简略的描述就能让你信以为真了吗。”

“感觉没人会拿这个来造谣吧。”

“嗯……”

~\\

确实,谁会闲到编出这种不存在的故事呢?

退一百步讲,尽管细节可能是错的,但肯定是有那么一件事发生了所以才传了出来,空穴来风嘛。


~\\

“可是诺啊,如果那是真的,咱们这样好吗?现场应该就在附近吧,那边肯定都被封起来了。这样,咱们在外面转一圈,远远看着就行啦。” 她没回答我,只是在那里对着我 “哎呀哎呀” 的。当然,你可以理解成撒娇,但我觉得这种行为只是为了说服我满足她自己好奇心的手段。

假设这件事是真的,那首先得到消息的一定是校内其他相关人员、安保人员以及同勤学堂的学生。此后才会像张诺午饭时问我那样传给无关人士。坠楼,怎么看都不是一件小事。如果这不是那位同学个人的意愿,那这毫无疑问是件恶性事件;反之,则是个人极端的选择。某个人极端的选择既是小概率事件,同时也是大环境的缩影。

张诺又是怎么看待这种事的呢?我不知道张诺是怎么想的,我只觉得张诺现在的想法不太好,毕竟我们和那件事一点关系都没有。现在如果进这座学院大楼的话,万一被逮住了,会不会有风险被当成涉案人员呀,接下来消息就会传开,八成到最后我们俩就被传成了凶手?
所以给各位声明一下,我之所以来这里都是张诺的原因,我是千百万个不愿意——我还有那么多作业没做,昨天还是凌晨 2 点睡的觉,哪里有精神陪她中午逛校园……如果不是她还算漂亮的话

长话短说,架不住她的折磨,我还是被她撺掇着迈入了惟心馆。

惟心馆真是不一般。踏入侧门,右侧看上去就极其现代化的吧台凳和蓝色弧形沙发椅瞬间让我体会到这栋大楼与某 “济心学堂” 的学院大楼之间的差别。我面对着的墙被很好地设计过,悬挂在墙面上的活动海报和赛事快讯令我对这座惟心馆的评分大大提高。唯一不足的是,一楼对我来说有点压抑,仿佛伸伸手就能摸到顶。但这也是惟心馆建筑设计的一部分吧。

“外面看上去一般,但没想到里面的装修很不错呀。”

“好像是前几年重装修了一次。所以我才经常来这里自习呀。”

哦,对的,刚才是听她说过。她还跟我说惟心馆还有一间跃层的阅览室,座位充足、开放时间长,最重要的是里面有电源。反观 “之心馆”、“开悟馆”、“德才馆” 等学院大楼,自习资源则是非常的稀缺。除了远一些,惟心馆好像没有任何缺点。

“这不是你们学院大楼嘛,你来肯定天经地义。”

“不过说起这个,诺啊,你作业做完了吗?”

“那五六门作业根本不在话下好吧。都还没开始做,但我很快就能搞定。”

“ddl 什么时候啊?”

“后天。”

“都是后天?”

“都是后天。”

“这么巧?”

“就那点小玩意儿,我随便写写就能应付完了。” 我听了她的话,心里说不出是什么滋味,她们专业那么轻松吗,真是隔院如隔山,还是说她太强了呢?不太清楚,我觉得她是要遭难了,但她自己还是看得很开,一边笑,一边带着我往里走。

“佳辛,你有没有觉得很奇怪?”

“我第一次来,这里有什么奇怪的,和平常不一样吗?”

“也是。我就是说今天太安静了。”

好像是这样的。惟心馆是一座学院大楼,午饭后肯定不至于一个人都没有的。更何况昨晚才发生那么大一件事,今天中午应该会有很多人有所耳闻赶来吧?哪怕没有闲人,按照我多年观看《名侦探柯南》的经验,调查员肯定也会维护好现场吧?我其实以为今天中午馆内会很 “热闹” 的。

“平常中午也是这样吗,诺?”

“也很安静,但今天格外寂静。”

我俯下身子,小声跟她讲:“你说,不会是整座学院楼都封了吧,然后咱们刚才是从一个小门进来的,那个小门没封。咱们还是快走吧……”

而未等我说完,张诺便笑了:“那不是什么小门,惟心馆一楼常开的就那么三个门,而且都那么大。倒是你,佳辛……” 她一把抓住我的胳膊,“你可不要乱跑了,我今天还指望着你呢。” 可我寻思我有什么好指望的,帮她完成作业吗?我只是个大三老废物罢了,能被高贵的张诺大小姐 “翻牌子” 一同用膳已经是我天大的荣幸了。

“咱们先说好了,我今天是被你带过来的。”

“好的,没问题,我向你保证。那我们第一步做什么?”

“嗯?什么意思。”

“就是调查呀!” 张诺停下脚步,一只手叉着腰,另一只手指着远方,以一种极其中二的姿势向我宣言她要查明事情的真相。

我大叹一口气,一开始我以为张诺只是好奇,顶破天来惟心馆里看一眼就走,没想到居然是想要办案吗?我承认,这可能是一件大新闻,但这明显不是我们应该掺和的。

“张诺,你认真的?”

“当然呀,不然来这里干什么。”

我真的很想说一句:“正常一点好不好”,但还是忍住了。她抽风就抽风吧,我想可能是作业太多弄昏了头,把脑子里的某根弦绷断了——如果她没开玩笑,几门作业一点没动,全在后天 ddl,很难不让人怀疑她的大脑已经来到了一个很 “微妙” 的状态。出于不想惹事的态度,我应该事事都顺着她来。或许过了今天,她就会在作业大军压境的情况下转变成那个内卷狂魔张诺,然后 24 小时无休爆肝作业。

而现在,看着她那极其诚恳却好像还有几分戏谑的眼神,我是不知道说什么好。罢了,管她是不是认真的,我划水不就行了,什么都不干,出事儿的话,就把自己摘得干干净净。不就是演戏吗,我不是同勤人,但演技这方面我也是一流的:

“啊!诺呀!咱们要找出真相,第一步你觉得是要找到案发现场吗?”

张诺嗯了几声,朝我点了点头。

“完全不是的。咱们应该最后去那里……或者说不要刻意寻找和前往。因为啊,一旦进入了现场,咱们就会落于现场设下的圈套中!所以啊,首先,咱们应该先把整座惟心馆挖个底儿朝天!你在一楼,我去二楼,就这么定了!”

我像个演员一样说着不经过大脑的话,吩咐张诺自己一个人去疯。她看上去也是十分入戏,喊了声 “Yes, sir” 后就跑得不见踪影了。从这个表现来看,她或许也很满意我顺着她过家家吧。

~\\

这是我第一次来到惟心馆。既然来都来了,我一定要先考察一下阅览室中的情况,这不是为了什么 “破案”,而是为了给以后想要自习的自己找一个好地方。

黄渡大学的自习资源极其匮乏,能够提供大量自习空间的反而是图书馆。回想两年前的冬天,我第一次来到那座图书馆,那时的我被馆内的条件打动了。当时还就读于四平路大学的我从未想到图书馆可以有那么多的插座和空位。但谁承想,慢慢地,我开始厌倦图书馆的景色——无论何时都是一堆人埋着头做题,偶尔还有人看直播、玩游戏。

图书馆,这座黄渡大学最高建筑物,仿佛可以被拆分成两个部分,一部分是人类可以自由活动的区域,主要用于学生自习;另一部分好似被道行高深的法师施了法术,正常人难以察觉到那些区域的存在,主要用于放置一些让图书馆名副其实的东西。

如张诺所说,惟心馆一楼有将近四十个座位。虽然只有方桌才有插座,但无论如何,这都已经远超其他学院大楼了。不仅如此,阅览室出门右手边不远处就是茶水间和卫生间,可谓是提供了学生自习的所有条件。比较可惜的是,如果馆内能再来一个饮料贩卖机就好了。

我沿着阅览室右侧的楼梯来到二楼,二楼的空间本就要小很多,书架更是极大压缩了空间。我简单扫了一眼,那上面都是些二十年前的出版物了。我听张诺说 “济世楼” 中不曾开放过的软件学院阅览室里也放着过时的低质图书。这里只做一个猜想,不一定对,或许学院大楼们都是如此吧。在书架与落地窗之间,有十余个带插座的位置,张诺说她经常在最靠里面的圆椅上写剧本。想到这个的我还在她的 “宝座” 上模拟一会儿我眼中 “同勤第一卷王” 的中午。

落地窗外是天台。虽说惟心馆地面有两层,但二层大部分都是露天的,只有阅览室和另一侧的教师办公室区域被玻璃与钢筋包裹着。我走出阅览室,绕了一个圈来到了天台外围的栏杆前。

“跳楼的话,为什么会选择惟心馆呢?明明只是一个两层高的建筑。” 我自言自语道。
说起来,我们俩连准确的事发地都不清楚,只知道一个 “惟心馆”。惟心馆这么大,东南西北都可能是出事的地方。我还说什么 “不要刻意去第一现场”,哈哈,那也得知道才行呀。不愧是我校,消息封锁得很好,消息灵通如张诺,都只能获取到那么一丁点信息。现在的她应该像个没头苍蝇一样撞来撞去吧。

我沿着栏杆向西面走,走着走着就发现了不同的景象。刚刚张诺带着我从馆东门进入,惟心馆的西面我还没有看过。

“哦……这边原来是这样的。” 在惟心馆天台西侧,望下去看到的不是一楼的地面,而是负一楼的露天花丛;而且地面相对于负一层十分高,这导致从二层平台望下去像是从四五楼往下眺望一般;如果有跳楼事件发生在惟心馆,那只可能是这样了,但是不应该。

大概过了五六分钟,我看到张诺蹦蹦跳跳地到了负一楼的露天花丛旁,装模做样地考察四周的环境。我静静地看着她,我想她大概是知道我在二楼吧,但她没有理我。

她是怀疑这里吗。但我觉得有问题。

问题在于,我面前的这块地方完全不像案发现场,先不说这几处花丛十分的整齐,没有按压痕迹,花丛周围的地砖也是遍布灰尘,不像是被打扫过的样子;再者说了,如果这里是现场,那应该被封上才对吧,张诺怎么可能这么轻易就进去呢。

不管她,我得赶快利用利用这段空下来的时间,思考下课程项目的下一步计划——没人干活该怎么办呀。这个学期我们有一门 “信安原理课程设计”,没有任何指导,并且作业要求十分 ”混沌“,大家全都是丈二和尚摸不着头脑。

突然,一阵妖风袭来,我的衣服被吹了起来。

俗话说 “水能生风”,坐落在济海东北侧的图书馆,一年四季妖风阵阵。到了晚上,坐在九楼的我分不清那究竟是凛冽的风声,还是飘荡在图书馆四周的恶鬼的咆哮。惟心馆建在济海西南侧,想不到这座二层小楼竟也有不小的风。

我倚在栏杆上,吹着风。

~\\

“叮咚叮咚,叮咚叮咚咚……” 好像过了很久,我的手机响了。不好意思,我给张诺设了特别关心。

打开 OCIQ,张诺说她找到了一个 U 盘。

“哦,那可能是你们学院的人不小心弄丢的吧。可以,你还有意外收获。” 我回复她。

“你难道不觉得这是个物证吗?” 她问我。

“差不多得了。要是物证的话,别人办案的早就收走了。哪里轮得到你。” 我有点绷不住了。

“不,这就是物证。我们阅览室二楼集合!” 她在花丛旁发完这条消息后就消失在我的视野中了。我拿她没办法,也就回到了玻璃房中。

话说今天整个阅览室一个人都没有,我抱着 “陪你玩就玩整套” 的心,看着张诺把捡到的 U 盘插到她电脑上。她一开始还想插我电脑,抱歉,我可不会让我的电脑插上这种来路不明的 U 盘,谁知道那里面有没有病毒。

~\\

“咋样啊,诺大侦探。” 我嘲讽她。

“好像确实有问题呀,这个 U 盘里面什么都没有。”

“是不是隐藏文件啊。”

“你可别把我当成一点电脑都不会用的人。”

不是隐藏文件的话,应该不会有人丢掉一块空 U 盘吧?我也不知怎得,可能是我体内济心人的基因作祟,鬼使神差地来到张诺旁边,示意她一边待着。后来回想,其实我也没做什么,只是简单检查了一下磁盘情况,看了看内部的数据结构。不要问我是怎么操作的,我就是胡乱敲了一通,抱着随便玩玩的心整出了一个文件,文件名是:“everything\_you\_want\_to\_know.exe”。
张诺有些惊讶地看着我,我得意地翻了翻白眼,撅着嘴给她说:“只是文件被删除之后没有多余操作罢了,这种恢复有手就行。”

“你好厉害啊,好厉害呀,你果然有用!”

“别别别,这个真的很简单,不要嘲讽我。还有,你别在意这个 U 盘了,该做作业了。”

张诺一把抢过自己的电脑,不熟练地划着触控板,就要去点开那个 exe 文件。我急促地叫了声她的名字,喝住了她:“这是别人的 U 盘,你怎么能随便看别人的文件呢?”

“这不是别人的文件,这是那位同学的呀。”

“别闹了,诺。” 皱着眉,我严肃地说道:“这只是你随便捡到的一个 U 盘。诺,6 个 ddl,你该做作业了。”

“你看这个文件的名字,明显就是留下来的线索呀。”

她说得没错。这太邪门了,我恢复文件过后那一瞬间就觉得不对。但没办法,已经被张诺看到,收不了手了。其实,我本来想的只是在她面前炫耀一下刚刚学到的东西,谁能想到成功了?又有谁能想到这个 U 盘还真的像是昨晚留下的 U 盘?我有点冒冷汗,小臂和脖颈的汗毛都竖起来了。

她笑了,见我不说话,反过来抨击我率先恢复 U 盘中文件的事实——“还不是你先,啊,整出来的文件,如果没这个文件,我想看还不知道看什么呢。”

我无言以对。只得认可她的行为:“这样,如果你打开之后确认是你同学的 U 盘,可一定要还给失主。哦对了,在那之前还要把文件删掉,以便恢复原样。”

“可如果这个 U 盘确认是昨晚那位同学的呢。” 她平静地说着。

“好啊,如果这个 U 盘是那位的,咱们就等着调查人员回来,把 U 盘交给他们。”

“你可真没意思。” 张诺又向我确认:“嘿嘿,那我就点了哦。”

“跟我没关系哈,都是你自己的行为,我劝阻过了。” 我侧过脸去。其实我还是有些好奇,在初中一年级的时候,我就觉醒了八卦之魂,一个下午的时间便挖掘出了班里大多数人的 “百度贴吧” 账号。我美其名曰:“对未知的渴望”。所以我并没有完全侧过去,余光还是瞟着张诺的屏幕。只见她的屏幕上弹出了一个窗体,上面好像浮出来几行字。

“佳辛,你来看看吧。”

“呃……不看。” 但我其实一直在看,只是看不太清楚。

“可以证实了,这就是那位同学的 U 盘。这里面装着的,一定就是想要传达的信息。”

我有些后悔,但又有些期待,缓缓转过头,盯着屏幕读了起来:

~\\

“你好!我也不知道最后会是谁发现这个文件,我也不清楚会将这块 U 盘交给谁。”

“但我可以确认的是,我将不久于人世。”

“为此我用 Electron 写了一个 app,将我想要说的话放在其中。”

“这就是我做的最后一个项目了吧。”

~\\

\section{第二个问题}

“毫无疑问,济心的。” 我说。

“Why,佳辛。”

虽说有些武断,但我还是自认可以下这种断言。"Electron"、"app"、“项目” 这些词,八成只有济心人或者经常接触济心学堂的人才能说得出来。为了给无聊的数据结构与人工智能作业做可视化,为了给老师打工或打某些无聊但有用的作品赛,济心人大约会在第三到第四学期接触这些东西。在自顾自地给张诺解释了一通后,她似懂非懂地点了点头。我不管那个,继续读了下去:

~\\

“如果你想知道一切的话,那就请回答我三个问题吧。”

“每回答正确一个问题后,我都会给出下一个问题的提示信息。”

“每个问题只有三次机会,次数耗尽后该应用会消除所有文件。”

“第一个问题:‘我的名字’。”

~\\

“佳辛,既然你确定这个人是济心的,那么你或许比我熟悉才对。” 看完窗体中的文字,张诺对我说道。

我承认她所说的没错,如果这个人真的是济心学堂的人,要么是和我同一年,要么是和张诺她一届的,至于毕设的学长学姐们……不太可能吧,都临门一脚了,本科应该还不至于。我解开锁屏,又看了一眼我的空间,可仍如先前一样,没有新的信息。

这就是仅有 270 个好友的我吗。

但也不能这么说,我的好友基本上是计算机科学与技术以及信息安全专业的同学。虽然说用 “Electron”,但万一是又写 Verilog,又写前后端,还炼丹的全能自动化呢?那我可就不认识了。

“诺,你也知道。” 我想她委婉地表达了我好友数量太少的现实,她也是会意。

“哎,在这方面还是得看我呀。”

张诺掏出手机,自己摆弄了起来,不久后,就给我 OCIQ 发了一张截图:“佳辛啊,你知道这个东西吗?”

我打开一看,从界面看上去,那好像是个我从来没用过的即时交流软件中的群组,群组名叫什么 “黄渡胶囊”。

“这 ‘黄渡胶囊’ 是什么啊?” 我问她。可谁想到这 “黄渡胶囊” 四个字说出口,她就是一阵大笑,不知道我说错了什么话……不过她笑得真好听。

“说得对,‘黄渡胶囊’。‘黄渡胶囊’ 就是你们济心学堂的学生在我校 ‘黄渡云’ 微博功能阉割后创建的群组。你呀,怎么连自己学堂的东西都不知道,哈哈哈。”

这样吗?“黄渡云” 被阉割功能好像还是因为某个人还有我们学校由于什么原因火了,嘛,在我看来就是些无所谓的事,对于我这种学生来说,只知道微博功能被关掉这个事实,从此生活少了一大乐趣。可是,那件事满打满算过去了两年,那个时候连学还没入的张诺竟然都知道吗,我不免对全知全能的张诺更敬畏了。

“不管那个,仔细看内容。” 张诺说。

~\\

“部分课程设置不合理!内卷难辞其咎!”

“我心无法平静。”

“你院终于卷出事了,大学俨然成了一个养蛊场,通过制造各种标准来选出最卷的蛊王,给予他们奖励——推免名额。哎,只是为 lkx 同学感到悲伤。”

“4 月了,风洞中心旁的樱花开了。”

“不会吧不会吧,惟心馆才两层楼,这怎么可能呀?肯定另有原因,有的人肯定懂我意思了,不懂的也别想懂了,知道太多没好处,多的我不好说,我擦。”

“真心希望校方能给出一个合理的解释,也给我们一个交代。”

“能不能别带节奏了,不要因这件事被网上心怀恶意的人钻了空子。”

……

~\\

看了这些评论,我没什么想说的。没办法,统计学规律在样本容量达到一个水平后就会生效,无论我在哪里,都逃不过。

“虽然评论区基本没几个正常人,但还是有些信息的。” 看完了 “黄渡胶囊” 中关于这件事的评论,我说道。

“没错。lkx,咱们两个可以在软件里面搜一下。”

“是林凯茜吗,是她吗,我们信安的一个学生……” 我通过首字母联想到熟悉的名字,下意识地说了出来。而张诺也把搜索结果直接举给我看,果然也只有这一个人。

“佳辛,你原来认识她吗?” 张诺平静地看着我。

我叹了一口气:“不能说是认识,只是见过一面、在 OCIQ 上聊过几句天罢了,她问过我一些关于培养方案的事情。”

“那你对她有了解吗?”

“很抱歉,并不。虽然见过面,也聊过天,但是并不深入。”

“没关系没关系。咱们赶快把她的名字填上去。” 张诺放下手机,在我还没反应过来的时候就打下了林凯茜的名字并点了 “确认”。如我们所想,昨晚在惟心馆选择了结一生的同学正是我的这位学妹——林凯茜。

~\\

惟心馆这起事件的中心竟是我们专业的人。

那种难以言说的感觉再次涌上了我的心头。不知不觉间,我没有再把这件事当作虚无缥缈、于己无关的传说,我开始了我自己的猜测。

一切都要讲究个所以然。如果凯茜先前存在精神上的问题,那我们就要考虑其精神疾病的形成原因;如果凯茜是出于感情上的矛盾,那我们就要了解她的家庭、室友,尤其是亲密关系。

但我本人更是倾向于她的学业压力。由于黄渡大学实行 “大类招生” 制度,刚入校的学生不能在第一时间确认自己的专业,而是在一年后,才在学校内部通过专业分流正式选择,就等同于将填报志愿的时间从高考结束推迟到了大一结束。听上去很美好,但这么做也存在很多问题。

首先是课程安排方面,原本分布在三学年的专业课被压缩到了两学年中。就以信安第三学期为例,导论、数据结构、数字逻辑、离散数学、信安数基、概率论、普通物理,其中大部分人还要多补一门程序设计。遥想我们这届第一学期时,单是 “导论” 和 “程序设计” 就够我们喝一壶了;在第三学期令我们哭笑不得的 “数字逻辑” 再次迭代了难度;原本放在第四学期的 “信安数基” 又挪到了第三学期,这就不难理解有同学因为课程压力过大而自暴自弃了。

其次是课程教学方面,“信安” 全称 “信息安全”,从理性、中立、客观的角度来看,我院教不了什么 “信息安全” 知识,黄渡大学此前也不具备特别舒适的环境。然而这个领域入门门槛不算低,我很难想象一个在第三学期初连 C 语言都写不利索的学生可以在第六学期前成为一个 “信安人”。据我了解,即使是原本三个学年的学习,从该专业毕业且此后以信息安全作为工作或研究方向的学长学姐也还是相当少。

最后是年级氛围方面,济心学堂分流采用 “择优录取” 方式,尽管不是直接使用绩点,绩点也还是很重要。二者共同促使同一专业下学生前两学期绩点方差较小。与此同时,出于大家都清楚的原因,存在相当一部分同学希望通过前六学期的成绩,在那个暑假获取推免名额。推免名额也是 “择优分配”,其在济心学堂的评价体系中,绩点占比极高,这让不少同学开始了内卷。从结果来看,“计科” 以恐怖的 “4.5 以上比例” 一骑绝尘,傲视全学堂其他所有专业。以年级均绩极高为基础,不少专业还将手伸向了竞赛加分与科研加分上。

简单来说,就是这样可以延伸出很多话题的三个方面。

然而,无论如何,问题出在学业方面只是我个人的猜测,我也不敢一口咬定。如果我要知道真相,我只能解开凯茜剩下的两个问题吧?

“诺啊,第二个问题是什么?”

“不太好办,刚才让你看你在发神,我就没念,总之你自己看吧。”

~\\

“第二个问题:‘我最好的朋友的名字’。”

“提示信息:127.1.0.1”

~\\

窗体中是这样显示的。我没有特别在意问题,反是被凯茜的提示信息吸引了注意力。“127.1.0.1”,任何济心人的第一反应都是 “一个不常用的回环地址”。但问题在于,这串 IP 地址又并非公网 IP,是要登录凯茜的服务器吗?这之中到底蕴含着怎样的深意?凯茜究竟想向我们传达什么讯息?单单凭借我的大脑,或许不太能得出结论。可是,难道要让我指望旁边正在百度 “127.1.0.1” 的张诺吗?

“诺,你知道这个提示信息是什么意思吗?” 我问道。

“你问我吗,我还想问你呢。一开始我就觉得好像和计算机知识有关,刚刚搜了一下发现它好像是一个 ‘IP 地址’。这方面知识我是一无所知呀。”

听了张诺意料之中的回复,我意识到 “依仗提示信息得到答案” 的这条路目前或许走不太通。“127.1.0.1”,“127.1.0.1”,哪怕它真的和 Web 方面的知识相关,我也没有那种能力可以找出线索。我虽在信安专业,但搞笑的是我们专业在第六学期,也就是这个学期才开始学习计算机网络。我深知,作为一个对学习有些追求的大学生,不应该被学院制定的无厘头培养方案推着走,但人的精力终究是有限的,我的计算机网络基础和实际应用能力非常薄弱。

如今,只能寄希望于凯茜留下的 “127.1.0.1” 并未使用 Web 方面的知识了。

“佳辛,你说话呀,你是信安的,这方面的知识应该很熟练吧?”

“你这句话说得我很心痛。我确实不会,我只对这个东西有非常粗浅的了解:RFC 6890 中对此有一些定义,127.0.0.0/8 都被定义为环回地址,都表示当前主机。我也只知道这些。”

作为一个当代大学生,没有实践经验仿佛是理所当然的事情。

“听不懂,完全不知道什么 ‘回环地址’、‘阿弗谁’ 是什么东西。”

我们俩又陷入了沉默。

玩过 “密室” 的人都清楚,破解谜题靠的是 “线索”、“直觉” 与 “推理”。“线索” 是思考的基础,浅层次的 “直觉判断” 与 “推理” 都是 “线索” 不充足的结果;“直觉” 是一个人天赋的体现,其精准程度决定了能少走多少弯路;而 “推理” 是建立在 “直觉” 与逻辑之上的系统方法,归根结底还是靠 “直觉”。

现在的情况是,我们第一时间无法得出一个有效的解决方案,继续空想下去也难以有所突破。正如 “密室” 玩不下去了就要向运维人员申请 “提示” 一般,如果能获得更多信息就好了。

“诺啊,咱们坐在这里也不是办法,到处走走看吧。”

“哟,怎么变积极了?” 张诺笑了,或许是听到我主动谈论这件事。

“嗯……虽然我不认识林凯茜,但她毕竟和我是一个专业的。相较于这个学校中的很多人,我和她的距离其实很近。说老实话,我仿佛觉得发生在她身上的事,就好像是发生在我身上一样。” 我自诩具备一定的同理心。

“真是令人感动啊。不过佳辛你并不是凯茜,再能将心比心也无法切身体会她的处境。” 张诺说完就站起身来,“走吧,咱们下楼看看,刚才我并没有观察过特别多的地方。”

“是啊,说不定还有新的线索。” 在询问张诺惟心馆是否安全后,我把背包和电脑包直接放在了地上,跟着她下了楼。

~\\

由于刚才直接奔向了阅览室,根本没有走过其他地方,我就拜托了张诺带我转一圈。非线性编辑机房、教师办公室、学生会办公室,就连馆外我也通过窗子看了个遍。今天的惟心馆果然奇怪,整栋大楼空荡荡,除我们二人之外,一个人影都看不到。

“确实不对劲呀,诺。”

“可能碰巧老师们都在开会,学生都在休息吧。”

“你不是想查明真相吗,今天这样的环境算是个不错的机会。”

“为什么这么说?”

“因为现在没有其他人,咱们做什么都不会被发现。”

“哈哈,怎么佳辛你做事都要偷偷摸摸的呢,大可以大胆一些。哪怕被别人看到也没关系。”

“这样吗。”

“你潜意识中还是想隐藏自己的行动,这是不可能的”,张诺指了指不远处的监控摄像头,“你难道觉得这个玩意儿是摆设吗?”

我倒是忘了这件事。监控,监控,现在除我们之外没有其他人,我们可以去监控室啊!

“哎呀,你倒是提醒我了,咱们去监控室吧。刚刚看过了,那里面没有人,趁这个时间进去看看昨天发生了什么吧!”

听到我的思想突然如此激进,张诺也是笑着点头,拽着我的胳膊就走:“说你一下就变得这么大胆了吗,好呀,监控室在北门。”

~\\

与济心学堂的之心馆相同,惟心馆的监控室也在一楼,而且也在一间大门旁边。我寻思这个时间,负责监控的保安不在监控室,应该是因为去吃午饭了,可再怎么说也不能不锁门就离开呀。不过也多亏了惟心馆安保人员如此低下的防范意识,张诺和我才有机会潜入监控室中。

走进监控室,我二话不说直接坐在中央的电脑桌前。

“佳辛,你已经进去了,但你知道这意味着什么吗?” 张诺跟在我后面,看到我一屁股坐在椅子上,以一种微妙的表情小声问我。

我大概可以理解她的意思。无论是捡到凯茜的 U 盘,还是私自操作磁盘恢复文件,这些事都是天知、地知、张诺知、我知的。只要我们两人不主动坦白,没有人知道我们在做什么。也就是说,先前我们的行为都是存在可解释余地的,通过监控很难准确捕捉到我们的所作所为。反观现在,我在惟心馆 48 个摄像头的注视下,大摇大摆地走进了监控室。要是被发现,这就不是一两句话可以搪塞过去的了。

“当然。” 我对她说,“这里是你的学院,如果你觉得不太好的话,可以退到外面。”

“无所谓。” 张诺也并不在意。她走到了我旁边,指了指显示器,让我操作下去。

我在心里佩服张诺。你如果让我进入其他学院大楼的监控室,我肯定有恃无恐——被发现又怎么了,我又不是他们的人。可你如果让我潜入之心馆监控室,哪怕借我一百个胆子我都不敢。

~\\

我之前以为需要猜主机用户和管理平台的两套密码。幸运的是,在我坐下时,中控电脑还没有熄屏,前者也就不需操心了。点开监控管理应用,果不其然,我对着密码框尝试了几个经典 “弱密码” 就成功登陆了管理平台,整个过程行云流水。

“嗯?你怎么知道密码是这个。” 张诺又用先前那种看到我成功恢复磁盘文件的眼神看着我。

“啊……几个弱密码罢了。你的笔记本密码可不要设置成一串纯数字和字母呀。现在……让我计算一下……”

昨天是周五,属于正常行课期间。据我了解,下一级信安周五 10 点到 11 点 35 分这段时间是有课的。假设凯茜中午就来了惟心馆,而且她是去食堂吃的饭——排队 25 分钟,吃饭大概需要 10 分钟,骑车到惟心馆只要 5 分钟左右,也就是 12 点 15 分进门。我就不妨先把时间线调早点,调到 11 点 50 分。

不得不说,这个监控平台还蛮难用的,我花了三两分钟才理解如何操作。

~\\

调好监控回放后,我搬了张椅子给张诺,她就坐着和我一起看起了监控。

考虑到张诺是同勤学堂的人,倘若保安在我们看监控的时候回来,那就不好解释了。我,本着为张诺感到心虚的态度,16 倍速高强度播放监控视频。北门通道中,人来来往往,一对对情侣搂搂抱抱,三两个女同学在门口抽着烟,职工带着孩子游玩,而我一个认识的都没有。好不容易看到一个熟悉的身影,我下意识喊出了声:

“欸欸欸!”

“怎么了?”

“切回二倍速,往回倒半分钟,我好像看到个熟悉的身影,可能是我们学院的人!” 我激动地对张诺说。但我忘了她根本不会操作,所以还是我来调的。

这次激动的结果令人非常尴尬,当我睁大眼睛紧盯二倍速监控时,却发现是张诺。

“啊这,这不是你吗?” 我有些无语,“所以你昨天中午 12 点 25 分来学院做作业,今天还有 6 个 ddl 吗,你在干什么?”

张诺笑着摆了摆手,让我不要哪壶不开提哪壶:“昨天其实也有个 ddl,我来惟心馆赶稿,学生会那边要的。”

好吧,我现在的心思没在她身上,继续看起了录像。将近一分钟后,监控来到了昨天的 12 点 39 分,我眼前又出现一个相对熟悉的身影,这次确实是林凯茜了。从时间来看,她从食堂出来后还回了趟寝室。哎,不愧是大二的学生呀,中午吃完饭回到寝室还能再有心思出来学习,曾经的我也可以,现在却是做不到了。

“佳辛,追踪一下凯茜的动向,她昨天应该在一楼,我没看到她。”

张诺说得有道理,哪怕她不这么说,我也会那样做。接下来要办的,就是切到最高的 32 倍速,观察凯茜下午乃至晚上的动向。

然而,事情并没有那么顺利。监控录像在晚上 9 点钟时戛然而止,这时的凯茜还没有离开过惟心馆。滑到下一个画面,那已经是今早 8 点的了。我明白,这是办案人员取走了昨晚 9 点后的监控录像,我们没法通过监控获得能对事件产生决定性推动作用的线索。

“没办法呀,诺。感觉断了。” 我躺在椅子上,撇着头看着她。

“那就走吧。” 张诺站起身,把椅子搬回原处。

我也着手恢复电脑桌面,确认无误后才跟着张诺离开。本来我想把我们进入监控室的那段录像删掉,但惟心馆这个监控平台的界面设计实在是太差了,我没能自行领悟出对应的操作方法,只好作罢。

~\\

出了监控室,不约而同,我们两人都走得很慢。

我不知道张诺在想什么,我是在思考些不合常理的事。

回想起来,刚刚过去的一个小时还真挺出乎意外的——凯茜是我们专业的同学,可空间中却没有任何消息;办案人员没能搜查出的 U 盘被张诺发现了;U 盘中居然有死者留下的可执行文件;今天中午惟心馆安静得离谱,监控室毫不设防……还有很多谜团,还有很多说不通的地方。

“我说诺呀。” 我有一嘴没一嘴地问张诺,“你说凯茜她想死吗?”

“什么意思?”

“不觉得很怪吗?如果我一心求死的话,顶多写份遗书。凯茜却用 ‘Electron’ 套了个壳。你可能不知道,套壳还挺费时间的。” 大二的同学可能还没有特别丰富的套壳经验,凯茜的 “Electron” 甚至很可能是为了这件事现学的,“你说,凯茜是抱着一种怎样的心理呢?”

“我不清楚。”

“你觉得监控缺失的那段时间,凯茜都做了什么呢?”

“我不知道。”

“话说惟心馆阅览室周五是允许通宵的?”

“我没试过。”

“诺啊,你刚才不还想当侦探吗,怎么现在就摆烂啦?” 我撅着嘴,用我最惯用的嘲讽语气嘲讽她。

“我也想查明真相呀!可我什么都不知道!” 她突然急了,这倒是把我吓了一跳。

下一个瞬间,我自以为完全理解了张诺的感受:她一开始说了大话,而现在可执行文件是我恢复的;第一个问题太过于简单,以至于她没有舞台;凯茜又在第二个问题上用起了 “计网” 上的概念;看监控也没得出什么结果。合着她什么事都没干。张诺她这么优秀,心里一定很要强吧?如果连自诩老废物的我都比不上,她该多伤心啊。一开始,她可能只是把我当成工具人来用,而现在我逐渐占据主导地位,她肯定不会高兴。佳辛啊,佳辛,你怎么就不能提前个几秒钟理解得这么透彻呢?

想到这些,我开始安慰她:“我清楚我清楚,诺小姐您别动怒啊。您别看那个林凯茜,啊,在第二个问题的提示信息中用了什么 ‘点分十进制’。说不定那根本就不是 ‘IP 地址’ 呢。对吧?”

她翻了翻白眼,问我:“那你说是什么嘛。”

“我虽然现在摸不着头脑,但也有很多猜想啊。”

好吧,其实我什么猜想都没有,只能现场乱编:

“不就是 ‘127.1.0.1’ 嘛,我分开说,说到 ‘127’ 你能想到什么?”

“惟心影院。”

“嗯?什么?”

“惟心影院呀,‘127’ 不就是惟心影院。”

“为什么 ‘127’ 是影院?”

“因为小影院门牌号就是 ‘127’ 呀。”

哦,对呀。谁说 “127” 就一定是环回地址的前 8 位?“127” 可以是介于 “126” 与 “128” 之间的数字,可以是 2 的 7 次方减 1,可以是 1 楼第 27 号房间,可以是字母 “abg” 的对应数字,可以是某位同学名字的谐音。而在惟心馆中,“127” 就代表着惟心影院。

我暂时不清楚 “127” 是否还具备其他含义,但就目前情形而言,这可能是唯一说得通的解法。如果 “127” 真的对应惟心馆中的一个房间,那么后面的三个数字应该不难解读。

不过我需要进到惟心影院里面去,我要看小影院中的构造,我想要验证我的猜想。

“哦哦哦!我的诺,你提醒我了,那个 ‘127’ 很有可能指的就是惟心影院啊!” 我兴奋地对张诺讲。

她也是个聪明人,不难理解我的意图。

“那后面的 ‘1.0.1’ 该如何解释呢?” 她问我。

“这个就得进到惟心影院里面才知道了。” 因为小影院就在阅览室入口右侧不远处,刚才我们在一楼闲逛时也经过了那里。但今天是周六,惟心馆内也没有举办什么活动,惟心影院并没有开门。

“但诺啊,现在门是关着的。你之前说周二和周五开放,那咱们就下周二晚上一起来吧!” 尽管张诺在我眼中确有通天彻地之能,但这种事也没法让我抱太大期望。情商位于平均值之上的我意识到了好机会,一招借力打力,平淡而不失风度、内敛却不失内涵地邀请眼前的美女三天之后一起看电影。

可我万万没想到张诺 “哦” 了一声就走开了,只留下一句:“小问题,我就是管这个的,你等会儿。”

~\\

“女神,你在我心中的形象更加光辉灿烂了,splendid!”

踏进门,掀开门口的遮光帘,我迫不及待地迈入惟心影院。

眼前的观影区四四方方,没有以弧形安排座位,也不存在特殊的站区,从前到后一共 9 排,每排大概 20 个位置左右,整体坡度也比较小。

就是个平平无奇的影院。

话不多说,我直接走到第 2 排,坐在了最左边的位置上。

“佳辛,要歇会儿吗?” 张诺看我坐下了,也放下书包,侧坐在 VIP 座位上扭头看着我。

“不,我没那个意思……不对呀,是我的问题吗?那个,你过来一下,别坐你那里,坐我这个位置。” 觉察到一丝不对的我站起来,走到座位后面,示意张诺过来。

她虽然感到疑惑,却还是走到我这排。

“怎么样?看到什么没有?” 我问已经坐在我先前位置上的张诺。

“你要我看什么呀?”

“先不管这个,你为什么能拿到惟心影院的钥匙?”

“我是 ‘惟心影院’ 的工作人员。你是没在 ‘观影群’ 里,如果你在的话,就能看到我偶尔发起 WeChat 接龙了。”

“哦,这样。既然如此,那你应该非常熟悉这个房间吧?”

“算是吧,肯定比学院中大部分人熟悉。”

“那你现在坐在这里。” 我双手摁住张诺的肩膀,“能发现一些与平常不同的地方吗?”

张诺东张西望、摇头晃脑,可没什么结果:“感觉没什么不同。那现在佳辛你可以回答我先前的问题了吗?”

我本来以为张诺不会很容易理解我的脑回路,没想到她还蛮上道的。我的思路大概是这样——凯茜和我都是济心学堂的人,正如我一样,凯茜也可能在日常生活中使用书本上的抽象概念表达自己的想法。凯茜如果使用 “127” 指代惟心馆中的小影院,那很大概率会继续使用后面的 “1”、“0”、“1” 指代影院内的有序事物。

惟心影院中的观影区,在张诺眼中只是一排排座位,在我眼里可是一个二维数组。

如果可以这样推断,“127.1.0.1” 中的第一个 “1” 与 “0” 便是指代第 2 排从左往右数第 1 个位置。

“但最后一个 ‘1’ 如何理解呢?” 张诺问我。

“问得好。因为一个座位可以简单分为两种情况:有人和没人。我倾向于使用 ‘1’ 表示有人。也就是说,你只要坐在现在这个位置,就能发现线索。”

可是,我们两个人仍没有头绪。之后我趴在地上对周围进行了地毯式搜查,结果也不如人意。我又提出了先前的猜想:“很可能办案人员先咱们一步。尽管没有经过这样的推理,也还是找到了凯茜留下的东西。”

“那有没有可能,你想错了?” 张诺对我说,“最后一点好似有些跳越。”

“你说得也有道理,不过你可以拿出更合理的方案吗。”

由于客观因素的限制,以及主观因素的影响,事件的侦察已经进入了凯茜与我的世界中。假使第二个问题破解,凯茜也肯定会使用已经沁入骨髓的书本概念包装第三问的提示信息。像张诺这样的同勤学堂人,应该起不到没什么作用。

不过要声明一下,以上心理独白的有效期截止于我听到下面这句话。

“我可以!” 张诺中气十足地喊了一声。

“什么?”

“我不如凯茜和你这样会写很多种编程语言的济心人;你刚才的解释我能理解也是因为我碰巧用过二维数组;论计算机专业知识,我远不及你们。但我有最基本的推理能力,你想,有没有可能它是个三维数组呢?”

“不可能。” 我直接否定了她幼稚的想法,“如果是三维数组,第一个 ‘1’ 应该表示 z 轴方向,那样就不是 ‘127’,而是 ‘227’ 了。”

“你说得对,但这里是惟心馆。它还在 ‘127’ 中,但却不在一个平面上。佳辛,你转过头看看吧。”

~\\

我忽略了它。

~\\

我不知如何称呼,阁楼吗?还是应该叫中控室?

在第 9 排座位之上,明显有一个突起的部分。它是那么的明显,而我却没有注意到它——因为我先入为主的判断,导致视线都被眼前的座位吸引了么?

惟心影院放映电影并不通过屏幕直接播放,而是使用投影。设备都放在那个中控室中。张诺跟我讲,当她预约不到自习室的时候,还会爬到中控室上面参加面试。

“诺大天才,以我对正确答案的敏感,我觉得靠谱。”

“那我们先上去再说吧,你跟我来。” 张诺蹦蹦跳跳地来到最后一排,等都没等我就直接开始爬铁悬梯。

不愧是惟心影院的运营人员,身手十分矫健,眨眼的功夫就钻进了中控室。可气的是,或许是她刚上去之后看见我还在下面晃荡,居然开口问我是不是胆小不敢爬悬梯。哎,这个张诺太不懂事,丝毫不在意我的考虑,弄得我很尴尬。我虽没有这方面经验,稳稳当当地上个楼还是不在话下的嘛。

上面那个自然段最后两句话是我还没爬梯子时的想法,事实是我费尽九牛二虎之力才爬了上去。

~\\

中控室很小,如外面看上去的那样小。

一张桌子,两张椅子,一个空调,一间不知道通向哪里的小门,还有几个装满物件的纸箱。张诺盘坐在地上,正在纸箱中翻找着什么东西。

“你这袜子不要了吗,怎么坐在地上翻箱子?”

“不重要,就这么大的地方,你从里面开始找,我从悬梯口开始。”

好吧,那我只好进到最里面待着。

张诺说的有道理,中控室面积不大。如果她对 “1” 的判断没错,那直接查找起来的时耗也很低。但我还是打算按照凯茜留下的信息找线索,因为一旦张诺的想法也不对,那我就有充分的理由确定这一事实。

按照刚才在楼下的假设,“127.1.0.1” 这个以点分十进制表示的提示信息,现在只剩下后面两个数字。我已经到了三维数组中的第 1 平面,这个中控室也应该有一个二维数组才对……

难道是我面前的这张桌子?

仔细一看,这张白色桌子上摆了两种设备:左侧的机器应该是舞台中控,我去年冬天和院会同学在国会 101 办晚会时见识过;右侧的机器我不认识,我推测是现代放映机。

“127.1.0.1” 的后两个数字,对应第 1 排第 2 个座位。

那答案就在放映机吗?

我走到它跟前,大浪淘沙般找了起来。不过事情并没有我想的那么简单。桌面、放映控制设备的夹缝、抽屉、桌底我都详细地排查了一遍,但并没有我想要的信息。

站起身,我扫视周围,以本人一直以来都十分强大的对正确答案的嗅觉,我敢确认中控室不存在任何线索。看着不远处还坐在这全是灰的塑料地板上翻箱倒柜的张诺,我心底起了一丝怀疑。

“诺,你为什么那么执着于箱子?”

“哦哦,你要说数组,那我想到的肯定是并排放着的箱子啦。中控室中复杂性最高的就是这些箱子,我经常上来,对这些箱子的情况也是了如指掌,多了什么少了什么,我能很快反应过来。而像其他地方,正常人也能发现蹊跷之处,所以就交给你啦。”

好吧,不无道理。

箱子交给了张诺,放映机也没有线索,我的直觉又否定了中控室内其余地方,那剩下的只有中控室深处这道门背后的空间了。我问张诺:“话说这个门是通向哪里的?”

“杂物间,但里面没什么物件。”

抱着死马当活马医的心,我打开杂物间的小门。里面一片漆黑,好在门外有些亮光,本就有些夜盲的我,才能模糊地看到杂物间中也有两个桌子。什么都没想,我一个箭步迈向右侧,借着从中控室射入的灯光,在那个破旧课桌的桌膛中,发现了一个粉蓝色信封。

~\\

“张诺!”

听到我大喊了一声,张诺放下了手上的事,抬起头看着深邃的杂物间。我就站在门口,举着那个粉蓝色信封。

“张诺!起来!给我过来看看!”

她推开全是灰的纸箱,撑着塑料地板站起后,又拍了拍全是灰的手和裙子。全都打理完,才缓缓走来。

“佳辛,声音小点嘛,中控室隔音不……”

她和我的眼神都不错。我的眼睛虽然经历了小初高 12 年加两年半计算机学习的摧残,但现在仍然十分坚挺。张诺的视力也不赖,我从来没见她戴过眼镜。在现在这个时代,不戴眼睛的学生,其罕见程度不亚于国家二级保护动物。

所以,张诺刚走到中控室中间,就看清了信封上的字。

“看清了吧?看清了吧?我还没动过,你自己过来把这个信封拆了。” 我说完便往杂物间里走,再次遁入了黑暗。

~\\

张诺啊张诺。

她早就得知一切了吧?她应该什么都清楚。哪怕她有些事情没想通,现在也应该全都明白了吧?呵呵,而我呢?今天早上 8 点钟开始,还在床上躺着的我就入了她的套,被她糊弄得团团转。像个没头苍蝇撞来装去的反而是我。

信封被我拍在破旧课桌的桌面上。

我翘着二郎腿,坐在杂物间另一侧的椅子上,看着门外的张诺没精打采,走起路来又磨又蹭,迟迟都没进来。

“差不多行了,快点吧。”

而张诺仍是停在门口。

没办法,我拗不过她。哎,可能我就是这样温柔吧。我抄起信封走到门口,直接塞到张诺手中。

~\\

我该如何描述当时的心情?

有些疑惑?有些愤怒?甚至有些欣喜?或许吧。

不过我认为,相较于上述所有情感,当时,我的无奈更甚。

信封上写的是:

~\\

“给我最好的朋友——张诺”。

\section{曾经}

我是张诺。

来自于某座城市中的一间还算可以的学校。我也不能说它太好,因为和我一起上学的同学们都对它有着或多或少的不满;我也不能说它太差,毕竟我还考上了这所大学。

五年前的现在,我在准备中考。

那时的我,非常喜欢考试。不是因为我的成绩有多好,实际上我在当时并不特别出彩,而是因为考试就意味着休息。

可能你们有些不敢相信。至少在我们那里,平常的学习生活是十分单调的——教室、食堂两点一线的生活。在教室中,除了上课就是自习。上课有什么意思?初中的那点东西,老师一点我就会,即使是看书也能很快理解。自习又有什么意思?作业不过是机械地重复那些枯燥乏味且脱离现实的知识点,除了能应用于中考之外,一无所用。

我所在的学校格外注重形式。什么都要 “接轨”,管理上尝试与军事化学校接轨;作业上尝试与其他学校接轨;连考试都要与中考时间接轨。不过这也倒好,中午和晚上我能多玩一会。

所以我才这么说,唯有考起试来,我才有些休息时间。

~\\

还记得,那是中考二模的数学考试后,我靠着教室外墙,看着走廊上嘻嘻哈哈的同学们。

“张诺,你在这里看什么?”

有女生在叫我。我沿着熟悉而陌生的声音看去,原来是林凯茜。

一个和我当了将近三年同学,但从来没说过几句话的人。

我不免觉得有些尴尬。但好说歹说,她也是我的同班同学,我也就应付了几句,比如什么 “我在放松自己的大脑”、“刚才数学没做得好” 之类的。

“你还会担心成绩吗?” 她问我。

“会的,有一点。”

不知道是哪个人才首先提出的 “二模决定论”,我们学校是很信这个的。你没听说过吗?就是 “二模考的成绩基本上等同于最后大考的成绩”。我的班主任经常在班上给我们强调二模的重要性,听得多了,我也是深以为然——不然怎么办呢?我第一次读初三,不懂这些。

“没事没事,数学没考好没关系,后面还有好几科呢。”

“是啊,后面还有很多考试。”

我大概不会聊天,把天聊死了。这之后,是一阵尴尬的沉默。

~\\

“话说……你想过以后要做什么吗?” 她问我。

“想过,当个高中生,再当个大学生,毕业之后开始工作。”

“不是说这些,你想过以后读什么专业吗?”

“太远了吧,现在有人问我高中选什么科还差不多。”

“好吧,那你打算选什么科呢?”

“应该是文科吧。”

“你是文理科都能学得很好的人呀。我想去读理科。”

“理科吗?意料之内的答案。那我问你,你想过以后读什么专业吗?”

“我想去读化学。”

“为什么?”

“我感觉化学很有意思呀!”

“是因为你每次化学都能考得很好吗?” 好像是这样的,我记得林凯茜经常被化学老师表扬,每次考试也考得不错。但我觉得有些幼稚,我们没学几天化学科目,以我们现在的水平,可谓是一点都不了解化学。

“或许是吧……”

“那也挺好的。至少你喜欢它,喜欢就有动力,这就够了。”

“嗯。”

~\\

之后我们没说过话。

我不太清楚为什么。不过这也是很正常。

林凯茜没有在我们学校读高中,她去了个很远的地方,远到我听说一次就记一次,记一次就忘一次。

高中的事情很复杂,我过得不是很好。上到一半时,我突然决定换一条路,最后选了个我们班没人填报的学校和专业。

三年很短,但它过得很快。一转眼就在读大学了。

大学里,我脱胎换骨,像是变了一个人。又或者说,我成为了不少人心中的张诺。

就在第一学期的某一天,我一边吃饭一边刷着 WeChat,没想到我认识的大学同学在朋友圈中提到了林凯茜。

吃完饭后,我立即联系了那个同学,得知了林凯茜也读上了四平路大学的事实。

她心中已经确定方向了吗?她是想去读化学吗?她们学堂怎么分流?她所在的学堂中都有什么有趣的人?她从大一就开始科研吗?她们高中怎么样?她高中都经历了什么有趣的事?

我难以想象她的故事。

怀着这样的问题,我在 WeChat 上尝试加她好友。

这样就算是命运中的相遇吗?

我不太清楚为什么。不过这也是很正常。

她没理我。

一年后来到黄渡大学,认识了不少济心学堂的人,我这才知道她们用的都是 OCIQ。

在大学意外遇到初中同学,这种感觉难以言说。

更令我意外的是,没想到她原来是这么好的一个人啊。所以与初中不一样,我们乱七八糟地在线下聊了不少。我们开始一起吃饭、一起自习、一起有一句没一句地回忆初中生活,哪怕在初中我们根本不熟。

今年春节时,我还以为又要像 2020 年寒假时一样。但元宵节后不久就接到了确认返校的通知。回来后,我发现林凯茜她像变了一个人一样,性情发生了很大的变化。

我有些不理解。其实她的第三学期课业很重,即使那样的压力她都没说什么,仍是笑嘻嘻地做作业。为什么到第四学期,明明轻松下来了,却笑不起来了呢?

不懂。

我不太清楚为什么。不过这也是很正常。

直到昨天。

中午赶完稿后,我就出去拍片了。晚上九十点钟,还在和请来的男女主一起在北翔吃饭的我,看到林凯茜发的消息:

“张诺你现在在哪儿?”

“我还在北翔,有什么事吗。” 我回她。

“没事,就是有点想喝喜茶了,你能带一杯回来吗?”

“可以是可以,但我回的时间有点晚。”

“大概什么时间。”

“可能……12 点多才能回学校吧,要不你找找别人?”

“没事,我晚上一直在惟心馆,那就麻烦你了。”

“你要在惟心馆等我?”

“嗯。”

“是课件没看懂吗?”

“没有。”

“那是写项目遇到麻烦了吗?”

“不是,我已经写好了。”

“哦哦,今天是周五,阅览室应该关得很早吧。”

“不,现在还没人赶我。”

“好吧,我不是管阅览室的,我不清楚。你要是回寝室了,记得告诉我一声。”

“嗯。”

当时我只顾得吃饭了,没想什么其他的。等到我买奶茶时,才感到奇怪——为什么她不自己买一杯?非得要我带。

买完奶茶,时间已经很晚了,地铁早已停运。为了去惟心馆方便,我让司机停在了学校正门。那时已经是凌晨,惟心馆的西大门不开,我只好步行到北门。走在馆前的石板路上,向来喜欢东张西望的我看到了地上趴着一个人。

这吓了我一跳。

我壮着胆子走过去,可没想到趴着的人居然是林凯茜。

我没控制住自己,大叫了一声,这一下让她睁开了眼。

“凯茜你怎么了?” 我心里格外着急,凑到她身边问她。从当时的环境来看,我猜出她是从楼下掉了下来。或许是意外,或许是其他人的阴谋,也或许是她自己的决定。但那时,这些原因都不重要,我忘了好多事,只想听她说话。于是我趴下身子,把耳朵凑到她嘴旁。

“张诺啊……裤子,U 盘,收下它……”

她好像还说了其他话,但我听不清,只能勉强根据听到的词揣测大意。她,是让我把她裤兜中的 U 盘拿出来吗?

我起身看去,不过 U 盘好像不在她裤兜中,而是已经掉在地上了。我拿起 U 盘,把它举到凯茜面前,问她是不是这个,她动了动嘴唇。我看到她要说话,就又立刻趴到她旁边。

“不要给任何人……”

“好的,我明白,我一定。”

惟心馆里面有了声响。可能是我喊得那一嗓子,把保安惊醒了。惟心馆不大,纵使保安先前还在睡梦中,可能听不出声音的大致方位,但他们不出几分钟也还是能找到这里。想到这儿,我心里有些焦急,又继续问凯茜:

“凯茜,我不知道问什么,不过你有什么想跟我说的吗?”

她没再回答。

“不要把 U 盘给任何人”,这是她最后对我说的话。

看她这样,我告诉她不要乱动,随后就直接冲入惟心馆喊保安。

再之后呢?好像经历了很多事。保安叫了急救和她们学堂的老师,我跟着凯茜一起坐车到医院,得知最终结果后,还去派出所做了笔录。

我一夜没睡。

等到天已经蒙蒙亮了,才回到寝室。不过我没选择休息,而是拿起电脑,出校找了个没人的地方。我本以为凯茜会在 U 盘中留下许多想说的话,没想到 U 盘中空空的,什么都没有。

“凯茜……是她没什么想说的吗?” 我的确经常与她在一起,可我也猜不透她。

“又或者说……凯茜用了一些处理手法?”

心里这么想着,我打开了 OCIQ,向一个叫李佳辛的人发了条消息:

“话说中午吃教师食堂吗?”

\section{三选一}

我把张诺拉进杂物间,搬了张椅子给她坐。在那个封闭的小空间里,她向我说了很多,甚至还说了些从未向办案员透露的细节。

“原来是这样。” 我靠在椅背上,“但我还有很多想问的。”

“你问吧。” 张诺表示来者不拒。

“你知道你是她最好的朋友吗?”

“我不知道。” 果然,这也是张诺她在看到信封时,那样惊讶的原因吧。此时,我有些为凯茜感到不公平,张诺名声远扬、好友漫天遍地,甚至有 “黄渡谁人不识君” 的称号。凯茜将张诺视作最好的朋友,但张诺她本人呢?嘛,不过应该也不差,不然不会为此忙活一个中午,还叫上了我这个工具人。

“那个 U 盘是?”

“是凯茜交给我的,而非从花丛中捡到。”

“话说起来,那 U 盘你连办案员都没给,为什么允许我看。”

“很难解释……这其实是一个过程,每个时间点的理由都不一样。如果非要让我用一个理由回答两个问题,那就是因为我觉得你是自己人吧,很没道理是不是?”

“既然把我当自己人,先前为什么那样瞒着我?”

“这算什么,我本来想着一直瞒到最后,谁想中途露馅了。”

“诺啊,你这么说我很伤心的。”

“还有什么要问吗?”

“当然有。监控是不是被相关负责人员拿走了。”

“是的。”

“那在监控中有没有看到凶手?”

“没有。那时是凌晨,惟心馆周围一片漆黑,天台上的监控设备相当于被废掉了。之后还调过主干道上的监控,但没发现可疑踪影。”

“凯茜身上的其他物品有没有提供什么线索?”

“并没有。后面他们把 OCIQ 和 WeChat 聊天记录看了遍,说是没发现什么特别的。”

“好吧。我想问的最后一个问题……后面你在派出所得知她是自杀,对吗?”

“是。由于已经差不多定性了,所以你今天中午在惟心馆周围看不到任何调查人员。”

“那今天惟心馆如此寂静的原因是?”

“这个纯属巧合。”

~\\

我的 “讯问” 到此结束。

忘记说了,刚刚张诺拆开看过,信封中空无一物。也就是说,这个信封的唯一作用就是用于提供第二个问题的答案。保险起见,我们二人对中控室与杂物间展开地毯式搜索,直到张诺感觉实在没有其他线索后才决定离开。

在了解张诺与凯茜的关系后,我反而没有刚才那么激动了。

张诺和凯茜是老相识,是从初中开始就认识的同学。她们二人关系一定很好,但却又不是那样知根知底。昨晚张诺赶到时,只见凯茜趴在馆外,除此一无所知,甚至不知道在凯茜眼中她就是最好的朋友。现在张诺知道了凯茜的心意,这已足以告慰凯茜的在天之灵了。而张诺她自己,得知了这件事实,也应能化解些许她心中的悲痛吧?

而我也觉得这件事大功告成,虽然这么一整,我好像就是个局外人。

“诺啊,这都快下午了,你差不多该回去休息一下了。”

“也不差这几分钟啦。这后面不还有一个问题吗,临门一脚了。”

“是啊,这 U 盘就是给你特意准备的吧。想必凯茜藏了一些心里话吧。”

张诺拿起放在 VIP 座位上的书包,我跟着她一起又回到了阅览室。到了二楼,我一屁股就坐在书包旁的圆椅上,让张诺赶快把她自己名字填进去。

“诺啊,第三个问题是什么?”

“佳辛你快来看!不是自杀!不是自杀!”

“什么?” 听到她说的这句,我一下子就弹了起来,椅子还没坐热乎。

一步跨到张诺旁边,盯着屏幕念了起来:

~\\
~\\

“亲爱的朋友,我不知道你是谁。”

“既然你认识我最好的朋友,接下来的问题也肯定不在话下。”

“第三个问题:‘杀死我的凶手名字’。”

~\\
~\\

我念完,这段话念得我浑身的汗毛都立了起来。

完了完了,黄渡大学有杀人犯啊!

但这种念头充斥我大脑的时间并不长,因为我反应过来了——

“诺啊,不对吧?”

“为什么不对?这是故意杀人案啊。凶手掌握了惟心馆四周所有监控的监视范围,不但把凯茜约到了天台监控看不清的地方,还利用视野死角做到了来去无形。这个人他知道惟心馆周围一片漆黑,到了晚上,馆外所有监控都是废掉的,它们都是摆设,都是摆设啊。”

“不,你稍稍冷静地想下,凯茜如果是被他人杀害,怎么会提前得知凶手姓名呢?”

“哦,对呀。” 张诺恍然大悟,“抱歉刚才冲动了……那这第三个问题是什么意思呀?而且没有像第二问一样给出提示,让人摸不着头脑。”

确实如此,这一问的难度显得格外高了。

凯茜,她究竟是什么意思?

不过问题不大,我身旁的这位少女,可是被凯茜誉为 “最好的朋友”。

“诺,你不是凯茜眼中最好的朋友吗?想来你肯定知道她的意思吧?”

“嗯……我是有几个猜想。” 张诺低头沉思。

~\\

张诺是这样想的:凯茜所说的 “凶手” 有两种可能释义。

第一种可能,那就是指的在惟心馆杀害凯茜本人的凶手,但由于凯茜也不清楚究竟是谁,所以只是把她眼中最有可能伤害自己的人作为答案设置。如果是这样的话,我们并不一定要找出正确答案,而是要站在凯茜的角度思考问题,看谁对她持有最深的恶意。但我们认为这种可能基本不存在,黄渡大学没有什么深仇大恨。

第二种可能,那便不是直接杀人犯,而是间接伤害凯茜的人。这可能是凯茜的室友,也可能是凯茜的家人。如果是这样的话,我们就要摸清凯茜的人际关系。

第三种可能,或许凶手不是人,可能只是一个抽象、虚拟的事物,比如 “内卷”、“绩点”、“小组合作”,甚至 “黄渡大学”。如果是这样的话,我们就要从长计议了。

~\\

“诺啊,你分析得好,这种阅读理解能力,不愧是同勤第一。” 我面无表情地夸奖她。张诺的分析确实十分全面,如果是在考试的话,我想给她满分。但这是现实,不是考试。她这样什么都说,只能等于什么都没说。

“刚才整理了一下思路。佳辛,不说虚的,对于第二种可能,我心中确实有几个猜想。”

“哦哦,好呀,你快说!” 反复横跳是我最擅长的科目,我在心里默默为刚才的想法道歉。

“呃……但感觉也不好说。”

“诺啊,我懂的。虽说在别人身后议论有失风度,但现在也是迫不得已了。”

“不是,我不是那个意思。” 张诺摇摇头,“我想的是,我是同勤人,可能你更能了解她吧……”

“说什么呢。” 我打断她,“你可是她的初中同学,并且在大学阶段也非常熟悉,你们是好朋友。我相信你对她的认识,也相信你不会对我说出不利于她的话,不会做不利于她的事。比起我这种只见过一面的人来说,你不知道高到哪里去了。”

“说吧,诺。” 我补了一句。

她点了点头,从书包中掏出一张稿纸,开始写了起来。

~\\

我们不能以常理推断。

对于一个人,心情不好的原因大多情况下是一个庞大的综合体,掺杂了生活中方方面面的不快。但令一个人最终爆发的,始终只是有限个直接原因。

~\\

以张诺最近一段时间对凯茜的了解,能够作为第三问答案的嫌疑人一共有三个。

第一个是凯茜马原课上的小组组长。

从最近两年开始,黄渡大学为了排课方便,特别喜欢在通识课选课上限定专业。举个例子,我上个学期的毛概课,全班都是计算机类下的同学;回想一年半前我的马原课,我在课上还能认识来自全校各个专业的同学们。

然而思政课得优率是有限的,一个班撑死不会超过 40\%。如果班上还全部都是本专业的同学的话,竞争关系就变得格外明晰。

马原小组每组 8 个人,8 个人都摆烂的可能性微乎其微。凯茜所在小组的组长,雷娟,就是个无脑追求高绩点的人。听闻思政课当选课代表就能有相当程度的加分,她第一节课就坐在第一排的中间位置,在那 100 分钟内保持高度的警觉,只为在老师询问 “谁愿意当课代表” 时抢先高举自己的左臂。不仅这样,她为了奠定自己的优势,在组队时还主动出任小组长。

这其实都挺好。但问题在于,她的 “追求” 并未达到极致。雷娟同学仅仅满足于在表面上做到光鲜亮丽,当遇到 “小组作业”、“合作展示” 等栏目时,第一个甩锅的就是她。借着她 “小组长” 的头衔,她能把甩锅做到尽善尽美。

第二个是凯茜的室友。

上个学期,凯茜搬到了黄渡大学。黄渡大学以 “加强专业内各方面信息交流” 的名义,以专业内优先配对的原则重新安排了寝室。也就是说,凯茜的室友也是信安专业的。

虽然重组寝室也是以学生意愿为主,但在第一学年疏于社交的人也不在少数。像凯茜她们之间没有任何了解、稀里糊涂就与其他三个人组成一个寝室,已经不算最差的情况了。如果张罗不到四个人,那只能等待抽签,甚至打散重组。

据张诺所说,凯茜的室友是她心情低落的关键因素之一。

第三个是凯茜的班长。

虽说我们在教学方面与其他学堂还有些许距离,但在行政管理方面却是比不少学堂正规的。我就觉得我们这一级管理得很不错——老师们非常通情达理,而且在我们遇到困难时也会尽心尽力地提供帮助,我非常感谢他们。甚至还曾有学堂中的学弟,当面向我表达对我们这级辅导员老师的喜爱之情。

这里我表明一下自己的立场,我认为优秀的管理者可以培养出一个优秀的团队。如果管理者自身水平高、眼界广,多提供大方向的指导是具有非常大意义的。当然,所处的环境不同,就需要具体问题具体分析,学弟学妹也会有自己的想法。

刚刚扯远了,话说回来,下一级信安班的班长名叫仰建平。

我和他还算熟悉吧。对于他,我的评价是不好评价。

他这个人比较特殊,是我认识的为数不多到了大学还打心底里用高中心态看待 “班集体” 概念的人。在下一级推行 “德育分” 制度后,学堂内加强了对活动参与度的要求。换句话说,就是不少活动需要各班班长在班上拉人 “充人头”。班级活动、白工零活、各种讲座,不懂得拒绝的凯茜经常被仰建平当软柿子捏。

最近,凯茜的心情本就不好,他这么一烦,很有可能成为压倒骆驼的最后一根稻草。

~\\

“话说凯茜没有男朋友吗?” 我听完张诺的三个猜测后,好奇地问道。

“这个事情我没直接问过,现在有没有也不太清楚。情感上的事,也不好太多过问。我是认识她先前喜欢过的男生,但那都是一段时间前的事了。我感觉最近没有,我没发现她把大部分时间用在哪个男同学身上。”

“这样啊,那你说的这三个都很有可能啊。” 我说。

“但也都很没有可能,是不是?” 张诺好像看穿了我的心思。

我点了点头,打算用我极强的将心比心能力,体验凯茜的心情。

但张诺对我说:“想太多也没用,要行动。”

“怎么行动?”

“这三个人的电话我都有,我们打电话直接问。” 张诺摇了摇她的手机,一脸坏笑地看着我。

“你一点都不社恐吗?”

“我也没说我们要用真实身份呀。”

“哦?你的意思是?”

“就像演戏一样,扮演几个角色去问。”

“诺啊,咱们要问就直接问,没必要拐弯抹角的。” 张诺这么做,岂不是在欺骗别人?这种事可能会落人口舌。

“行,你就真人出演,去问你学弟。我打电话给前面两个人。我们两个在这方面互不干涉。” 张诺很得意。看着她兴致勃勃的样子,她恐怕是要整些花活了。

“我可不打电话,我在 OCIQ 上直接问仰建平就行了。”

“行吧,那你就看我表演吧。” 她妥协了,“佳辛,你等会儿尽量不要出声,我开免提,我们两个一起听对面的回复。”

我耸耸肩表示无所谓。张诺把手机放在木桌上,拉了一张椅子出来让我坐在她旁边。不一会儿,就拨通了电话。

~\\

“您好,请问是雷娟吗?”

“你是?”

“我是黄渡大学传媒与艺术学院大楼安保部门负责人,欧阳张诺。昨晚林凯茜同学在我院学院大楼惟心馆发生意外,调查发现你有间接致使他人伤亡的嫌疑。”

“啊?”

“还希望你配合我们的调查。不然的话,这将对你的升学造成极大的影响。”

“啊?不是。啊?别呀。这怎么就跟保研……我昨晚光在缝报告了。那个 ‘卫星馆’ 在哪里?我不知道啊。”

“不要废话。现在我问什么,你答什么,了解了吗?”

“了解了解。我真的什么都没做啊,林凯茜她怎么了?”

“现在才想起问这个?刚才不是说了吗,不要废话!”

“好的好的!”

我露出不可思议的表情看着一脸严肃的张诺。

“你和林凯茜是什么关系?”

“同学关系,我们不熟。”

“你和林凯茜最近有过交流吗?”

“有。”

“仔细说说。”

“这个学期我们一起上马原,哦,就是一门通识,通识课,三个学分。按照学长学姐的说法,这门课得优还蛮难的,课本上概念很多。这门课要求写期中论文,平常还有小作业,哦,还有 pre,不过还好,期末考试是开卷,不然背起来是真的难顶。我这学期一共才 20 出头的学分,要是这种和专业无关的课得良就太难看了,我还得给专业课预备出几个中良的位置呢,为了拿优,我们得……”

“谁让你说这些了?说和林凯茜最近的交流。”

“抱歉抱歉,马上说。我马原和林凯茜在一组,最近 pre 要轮到我们了,那个 pre 其实是加分项,不是一定要做,组里面意见不一,我就说我牺牲牺牲,我上去讲,然后让他们决定下,啧,呃,谁做 ppt,啊,最后他们合计了一下,让林凯茜做 ppt,其他人搜集材料。”

“之后?”

“之后我就给林凯茜提了点 ppt 上的要求,前面小组都做了 15 分钟的嘛,我就想我们组得做个 25 分钟的,也不是说我要讲 25 分钟,就是为了显得我们准备充分嘛,很多页都是不讲的,快速跳过去,但是那些也得好好做,除此之外就是模板啊,小标题啊,动画啊,其实也没时间放动画,就是要双击快进动画那种感觉,还有字体啊,这些方面的要求,好像也没其他的了,我就说了这些。”

“可以。李警官,和聊天记录中说得一样。” 张诺突然对我说了一句。

“啊啊,对对对。” 我懵了一下。

“聊天记录?哦哦那就好办了,我什么都没干啊,聊天记录为证,我也没任何理由,她还得做 ppt 呢,话说,您好啊,我想问下,她现在咋样了,还能做 ppt 吗?”

“应该暂时不能了。” 张诺顿了一下,说道。

“哦这样啊,我感到很可惜,您情况了解完了吗?还有什么要我做的吗?”

“没有了,再见。” 张诺直接把电话挂了。

~\\

“这是什么人啊?她感觉可惜个鬼!” 张诺非常生气,她转过头问我,“你们,你们专业都是这种人吗?”

雷娟?我没印象,可能是隔壁计科的?

“别这样说,我对那个人没什么印象。再者说了,你看我们计科信安不还有我这种人嘛。” 我尴尬地笑了笑,希望她不要因此对我们专业产生什么坏印象。不过我的解释太过苍白,张诺什么都没说。

坏了,不会在她眼里,我和那个雷娟本就是一类人吧?

“我看就是这个人了,咱们直接填她的大名。还有比这种人更过分的?” 我故意说道。

“呵呵,也未可知。估计她室友更过分。” 张诺冷笑了几声,继续打电话给凯茜的室友。

~\\

“喂,你好。” 张诺一改严肃的语气,轻轻地说道。仅仅是三个字,我却能从温柔轻软中听出一丝悲伤。

“喂,您好,请问能听清吗?”

“可以,你能听清吗?”

“我这边可以,请问您是?”

“我是凯茜的姐姐,我叫林凯思。”

“原来是凯茜的姐姐,经常听凯茜提起您,凯思姐姐稍等,我出个寝室。”

好家伙,这次张诺扮演的是凯茜的姐姐,角色切换之迅速、情感流露之真实、遣词造句之讲究,这就是惟心馆的同勤人?

张诺静静地等着,我屏住呼吸听着,电话那端传来了关门声,传来了急促的脚步声,传来了风声,以及逐渐响动起来的洗衣机声,凯茜的室友好像来到了洗衣间。

“姐姐好,我这里是不是有点太吵了,能听清我说话吗?”

“还好,能听清。还没问你的名字。”

“我叫冷清心。”

“那我就叫你清心了,好吗?”

“嗯。”

“你们应该刚回来不久吧。”

“其实也回来一段时间了,上午去做了笔录。姐姐,这真是非常令人伤心的一件事。”

“谢谢你,清心。我现在心情平复了一些,所以才打电话想问问我妹妹在学校是怎样的。你现在是在寝室楼吗?”

“是的,刚才在寝室里,现在在走廊。”

“真抱歉,打扰你休息了。”

“没有没有,我刚刚在写作业。其实是室友在联机,寝室声音有些大。”

“这样啊,另外两个室友在打游戏吗,你们寝室氛围真好。”

“不是,一个室友在打游戏,另外一个在睡觉。”

“哦……那凯茜平常中午会做什么呢?”

“她中午一般都不在寝室,偶尔回来我也有些没注意她在做什么。”

“清心,凯茜在寝室里怎么样?”

“我和她关系还可以,她和现在正在睡觉的颜宝语有些不愉快。”

“为什么呢?”

“其实是人生观上的不同。风语她有一套自己的三观,她们两个曾经绊了些嘴。姐姐你也知道,凯茜平常生活是很努力的。但风语她就比较……摆吧。其实都没什么错,两种人生观都有可取之处,大家情况不同,为这件事争吵其实有些可惜。”

“是啊,凯茜已经很努力了。”

“姐姐,那个,她和正在打游戏的唐雅莉,关系也不能说好。”

“因为作息原因吗?”

“凯茜原来和姐姐你说过啊。其实有这部分原因,但这是最近的事,这学期凯茜作息有些紊乱,身体也经常不舒服,晚上睡不着觉。其实雅莉大佬也挺刻苦的,总是后半夜写代码,不是写作业,是写她自己的项目。但是吧,她那个键盘的声音还是有些大的。”

“嗯。”

……

~\\

张诺和电话那边的冷清心聊了十几分钟才撂下,我是一点有效信息都没听出来。

但她说,这个冷清心的嫌疑更大。好吧,女生之间的事情我丝毫不懂。

看她那边告一段落,我就打算从 OCIQ 上发消息问仰建平,可没想到张诺拿过我的手机,直接拨起了 OCIQ 语音。

“诺啊,你怎么这么过分,赶鸭子上架啊!快把手机还给我。” 我伸手抢手机,着急地跟张诺说。

而她却淡定地让我把手机抢了回去:“好啦,这下你想挂对面也知道了。我劝你将计就计哦,佳辛。”

可恶,她说得有道理。

“佳辛,我稍后就在旁边给你打分,看你具不具备表演才能。”

无语。可无语归无语,我得想个说法啊。

~\\

“学长啊,是点错了吗?” 我的仰建平!你怎么能接得这么快?

“啊,那个……”

“快说话啊,别人等着呢。” 张诺在一旁煽风点火,离得这么近,她的声音都被录进去了。

“喂?学长?怎么还有女生?”

“那个……我是被别人逼着来问几个问题的。”

“哦?哦。哦!我懂了,大冒险是吧?”

救命了,仰建平,你可真是个聪明人!

“建平你别说出来,懂就行了。” 我就坡下驴。

“明白,学长。你想问什么。”

“话说你们班是不是有个叫林凯茜的女生?”

“对对,学长你?看上她了?”

“什么跟什么啊,不是。我这不……正跟她好朋友玩游戏呢吗!” 我看了眼张诺,非常自然而又不自然地笑道。

“明白,明白。”

“我就想问下,在你眼里,林凯茜是个怎样的人。”

“好人!绝对的好人!”

“好人?怎么说?”

“我平常和她交集不多,但她帮了我很多忙啊。学长你也知道,学校总是放下来很多任务,这个要上报人数,那个要上报比例,咱们专业报上去的数字,真的不太好看。”

“是啊,不过你在我面前说数字不好看,着实有点不妥当了——你比我积极多了。”

“凯茜就这方面好,无论我什么时候找她帮忙,她都来者不拒。上大学后,我没见过比她还为班集体着想的同学了。她不是班长都可惜了。”

“好人!”

“对,好人!”

“真羡慕你们班啊,又有活动办,又有钱拿,申了不少经费吧?”

“不说我还忘了,这些文书也有凯茜一份。经费是不少,但像一些和班集体有关的奖金下来得就很慢了。”

“和报销似的?”

“和报销似的!”

……

~\\

我和仰建平嘻嘻哈哈聊了三四分钟,最后以互道 “谢谢谢谢” 结尾。

张诺看我撂了电话,摇了摇头,对我说:“感觉什么都没问出来。”

“我也这么觉得。这三组人,好像都差不多呀。” 我无奈地笑笑。

“马原小组的雷娟、一个寝室的冷清心、颜宝语、唐雅莉,以及她的班长仰建平,我们选一个吧。”

“诺啊,就这五个候选人。虽然说已经很难选了,但是否还是太草率了?” 我试探着,小声地问她。

“是,所以还有些没调查的,比如做项目非暴力不合作的刘利、她高中时的暗恋对象邵梓昕、整天用凡尔赛文学搞人心态的蔡双秋、阴阳怪气传播负能量的网友辛珀萨缇……”

“等等,怎么还有这么多啊?”

“哎,因为她身边的那些人,令我有所耳闻的实在不少。”

在人际交往这方面,张诺哪里都好,就是认识的人太多。诚然,“认识的人多” 在大多数时刻都能带来绝大的优势,但到如今,却成了一项劣势。

我坐在木桌前。张诺又把电脑掏了起来。我一会儿看看她的电脑,一会儿看看她。看着看着,我突然想到了些什么。

\section{End}

糟了,莫非事情如我所想的那样?

“诺啊,我渴了,你能帮我买瓶饮料吗?” 其实我根本没渴,只是我心里想到了一些东西,我要把张诺支开,自己一个人做做验证。

“可是惟心馆没有自助贩卖机,我给你接瓶水行吗。” 不愧是张诺,对惟心馆的各项设施了如指掌。如果只是去接水的话,并且我的假想成立,时间应该完全不够吧?

“我杯子就在书包里,你去拿一下吧。我这里稍稍有一点头绪,需要仔细想一下。” 我说。
她站起身,一路小跑到不远处的椅子旁,开始翻我的书包。不巧的是,我水杯中还剩了半瓶。“你水杯里面还有不少水呀,佳辛。”

“呃……但我真的很想喝 AD 钙……或者是劲凉冰红茶。” 说到这里,我眼睛死死盯着屏幕,尽全力掩盖这种不自然:“要不你帮我去图书馆买一瓶过来吧,反正也不算远。” 书包里还有水没喝确实令我有些尴尬,但这种情况在她跑向我书包时我就预料到了,所以好像也没那么尴尬。我的余光瞟到张诺面露难色,我想她这样犹豫,肯定是怕我在她离开时破解了凯茜的这一层加密。嗐,为什么她和我不是一个学院一个专业的,如果她知道我平常上课都在学什么,平常作业的完成情况到底有多差,那就能知道我到底几斤几两,根本不会把我想象成一个对电脑无所不知的伟人了。

“好吧,不过你可不许私自把密码解开。我,你要是有头绪了,一定要联系我,等我回来之后再说。” 听到这句话,我难掩激动,连声应付她,把她哄得服服帖帖。等到我不再能听到她的脚步声时,我又打开了那个熟悉的界面:

“第三个问题:‘杀死我的凶手名字’。”

再次看到这行字,我应该犹豫了一下。最终还是在那个输入框中键入了我心中的答案——“张诺”。

~\\
~\\

“P.S. 你最后看到的这篇,已经是我加工多次后的了。我想明白了,不是你的错,也不是我的错。这是我留给这个世界的最后一段话,而下面是我的终稿,也算是我向这个世界输出的最后一篇 ‘文档’。”

“我知道以你的性格,会陪我走完最后一段路。我也知道以你的聪明,或许会在看到这个问题之初猜到答案。请原谅我用这种方式伤害你。”

“你不会记得我们第一次见面是在哪里。因为我也不记得了。相比于你这样的社交达人,我的社交圈小的可怜,哪怕这样我都不记得那些细节,更不必说你了。短短的一生中,生前最为要好的朋友,其名字能够作为第二个问题答案的人,都没有那么重要,我真的不知道什么是重要的。”

“其实我知道。”

“对我来说,那便是 ‘活着’,是 ‘活着’ 的这份感觉,或是我确认自己正 ‘活着’ 的安心感。但当我不再能感受到我正 ‘活着’,当我失去了对生活的热情,当我听不到心脏跳动的声音时,那我便死了,或者我就该死了。”

“我能想象你会怎样安慰我——先是 ‘退一步海阔天空’、‘不要把自己逼得那么急’、‘大不了不做了’、‘摆烂就完事儿’、‘昨天我一点都没学’,用这些话麻醉我。”

“我会告诉你道理我都懂,你也自知这些场面话于事无补,所以又会说——‘其实拿优很简单,只需要平常好好做作业,考试前狠狠复习一个晚上’、‘其实还有其他途径,可以打两个比赛,现在还有时间’、‘其实绩点高什么都代表不了,只是方便拿奖学金’。”

“我会表示这些我也明白,我知道学习只是生活中的一小部分。可当雪崩真的到来时,我还是放不下。这时你会假装着急——‘是这样的,人与人是不同的’、‘绩点比不上就不比了呗,就摆烂呗,能行吗’、‘总有排在前面的人和排在后面的,有排名就有这样残酷的现实’、‘一辈子不就是不断向上攀登的过程吗’。你会用这样的话来刺激我。”

“我会告诉你我已经听了太多了,这时你露出一副怅然若失的表情,情到深处落下几滴眼泪,抱着我说——‘不怪你,你已经尽力了,不是你的错’、‘好好吃一顿,睡个觉,明天又是新的一天’、‘大家都会帮你的,心情低落是短暂的,负面情绪是会过去的’。”

“识趣的我也就不会再闹了。”

“我想,没有必要进行无意义的对话。无论是嘴上摆烂实际内卷的 ‘卷躺二象性’,还是被不少人奉为圭臬的 ‘高绩点无用论’;无论是为了让我平静思考而主动揭示的 ‘物竞天择,适者生存’ 观点,还是最后 ‘人生不易,冷暖自知,余生不长,善待自己’ 的苍白安慰……让我们跳过这些平日里的客套话,聊一聊它们之后的课题。”

“接下来请点开另外两份文件。”

~\\
~\\

果然是她。

凯茜告诉我,我猜对了。

同级目录下一口气产生了三份文件,一个 “README.md”,另两个分别是 “我.docx” 与 “last.docx”。

虽然这个答案是我想出来的,但我不免还是有些惊讶。在键入她的名字时,我确实有那么一点点,希望我能看到被加密在文件夹内的文件,可我又多么希望我的想法是错的,我宁可少了一次尝试机会,宁可被去给我买 AD 钙的她痛批一顿。

心中,痛苦、焦灼、无力、悲伤……无数种情感涌上心头,我趁这个劲头儿点开了 “我.docx”。

~\\
~\\

“去死去死去死去死去死去死去死去死去死去死去死去死去死去死去死去死去死去死去死去死去死去死去死去死去死去死去死去死去死去死去死去死去死去死去死去死去死去死去死去死去死去死去死去死去死去死去死去死去死去死去死去死去死去死去死去死去死去死去死去死去死去死去死去死去死去死去死去死去死去死去死去死去死去死去死交际交际交际交际交际交际交际交际交际交际交际交际交际交际交际交际交际交际交际交际交际交际交际交际交际交际交际交际交际交际交际交际交际交际交际交际交际交际交际交际交际交际交际交际交际交际交际交际交际交际交际交际交际交际交际交际交际交际交际交际交际交际交际交际交际交际交际交际交际交际交际交际交际交际交际内卷内卷内卷内卷内卷内卷内卷内卷内卷内卷内卷内卷内卷内卷内卷内卷内卷内卷内卷内卷内卷内卷内卷内卷内卷内卷内卷内卷内卷内卷内卷内卷内卷内卷内卷内卷内卷内卷内卷内卷内卷内卷内卷内卷内卷内卷内卷内卷内卷内卷内卷内卷内卷内卷内卷内卷内卷内卷内卷内卷内卷内卷内卷内卷内卷内卷内卷内卷内卷内卷内卷内卷内卷内卷内卷意义意义意义意义意义意义意义意义意义意义意义意义意义意义意义意义意义意义意义意义意义意义意义意义意义意义意义意义意义意义意义意义意义意义意义意义意义意义意义意义意义意义意义意义意义意义意义意义意义意义意义意义意义意义意义意义意义意义意义意义意义意义意义意义意义意义意义意义意义意义意义意义意义意义意义不不不不不不不不不不不不不不不不不不不不不不不不不不不不不不不不不不不不不不不不不不不不不不不不不不不不不不不不不不不不不不不不不不不不不不不不不不不不不不不不不不不不不不不不不不不不不不不不不不不不不不不不不不不不不不不不不不不不不不不不不不不不不不不不不不不不不不不不不不不不不不不不不不不不不不作业作业作业作业作业作业作业作业作业作业作业作业作业作业作业作业作业作业作业作业作业作业作业作业作业作业作业作业作业作业作业作业作业作业作业作业作业作业作业作业作业作业作业作业作业作业作业作业作业作业作业作业作业作业作业作业作业作业作业作业作业作业作业作业作业作业作业作业作业作业作业作业作业作业作业捷径捷径捷径捷径捷径捷径捷径捷径捷径捷径捷径捷径捷径捷径捷径捷径捷径捷径捷径捷径捷径捷径捷径捷径捷径捷径捷径捷径捷径捷径捷径捷径捷径捷径捷径捷径捷径捷径捷径捷径捷径捷径捷径捷径捷径捷径捷径捷径捷径捷径捷径捷径捷径捷径捷径捷径捷径捷径捷径捷径捷径捷径捷径捷径捷径捷径捷径捷径捷径捷径捷径捷径捷径捷径捷径项目项目项目项目项目项目项目项目项目项目项目项目项目项目项目项目项目项目项目项目项目项目项目项目项目项目项目项目项目项目项目项目项目项目项目项目项目项目项目项目项目项目项目项目项目项目项目项目项目项目项目项目项目项目项目项目项目项目项目项目项目项目项目项目项目项目项目项目项目项目项目项目项目项目人生人生人生人生人生人生人生人生人生人生人生人生人生人生人生人生人生人生人生人生人生人生人生人生人生人生人生人生人生人生人生人生人生人生人生人生人生人生人生人生人生人生人生人生人生人生人生人生人生人生人生人生人生人生人生人生人生人生人生人生人生人生人生人生人生人生人生人生人生人生人生人生人生人生人生文档文档文档文档文档文档文档文档文档文档文档文档文档文档文档文档文档文档文档文档文档文档文档文档文档文档文档文档文档文档文档文档文档文档文档文档文档文档文档文档文档文档文档文档文档文档文档文档文档文档文档文档文档文档文档文档文档文档文档文档文档文档文档文档文档文档文档文档文档文档文档文档文档文档文档爱情爱情爱情爱情爱情爱情爱情爱情爱情爱情爱情爱情爱情爱情爱情爱情爱情爱情爱情爱情爱情爱情爱情爱情爱情爱情爱情爱情爱情爱情爱情爱情爱情爱情爱情爱情爱情爱情爱情爱情爱情爱情爱情爱情爱情爱情爱情爱情爱情爱情爱情爱情爱情爱情爱情爱情爱情爱情爱情爱情爱情爱情爱情爱情爱情爱情爱情爱情爱情爱情爱情爱情爱情爱情爱情”

……

~\\
~\\

我惊呆了,凯茜就这样用五号字输出了 127 页,将近 22 万字的文件。

“报告”、“工作”、“实验”、“生存”、“荣誉”、“自习”、“组队”、“展示”、“社团”、“资源” 等等一系列我上了大学后才逐渐理解其中含义的词被她以一种毫无联系、毫无逻辑的方式陈列在这个文件内。我无法分析她在创造这份文件时究竟夹杂着何种情感。密密麻麻的字被整齐排列在规定好的条条框框里,黑一块白一块。这些词争着挤入我的瞳孔,疯狂地冲击着我的视网膜,视觉神经不堪重负,大脑的某个位置被迫刻下了只要我一息尚存,便永不消退的画面。我着实有些生理不适了。

关闭文件,我又打开这个目录下的最后一份。

~\\
~\\

“你们没有错。”

“我也没有错。”

……

“错在我昨天熬了夜没有赶上今天的早课,错在我关键时刻运行速度太慢的电脑,错在我在马原小组中做 ppt,错在我拖得太久最后实在不想做那门课的作业,错在我上报了误差过大的实验数据,错在我在图书馆找个不算差的位置花了太多时间,错在我闲下来的时间没有提升自我,错在我太过社恐不敢主动与他人交流……错在我看起来什么都没做错。我知道啊我知道,只要我想去做就能做,明明只要开始就好的,但我做不到。”

……

“当一件不好的事发生了,有 90\% 的人只看到了结果 / 只想看到结果 / 想让别人只看到结果;9\% 的人会主动考虑造成如此结果的原因。0.9\% 的人会瞬间得出正确的原因;0.09\% 的人反思自己的过去并做出积极的改变;0\% 的人能回到过去,阻止这件事发生。”

……

“这该死的环境,我该怎么做才好?过上自己想过的生活,这很难吗?不,我根本不知道我想要什么生活。我必须过得比别人好吗?并不,但我能放下我好胜好斗的内心吗?我会心甘情愿看别人比我优秀吗?哪怕都这样了,我根本学不会摆烂,我只能被这该死的环境裹挟,他们在做什么我就跟着做什么。这鬼生活什么时候才能结束?”

……

~\\
~\\

是一份万余字的文件。这份文件的内容有些散乱,看上去是 “日记”。

我心中没有波澜。

我该做什么?我能做什么?

不知为什么,我的手动了起来。那个状况下,我或许做了一件错误的事:退出文件预览器,我删除了这三份文件,同时也故意输入了两个错误答案让这个可执行文件报废。

一切准备妥当后,我望着窗外的天台,发愣。

~\\

“佳辛,我回来了。”

哦,是张诺回来了。随着一阵急促的脚步声,不一会儿她就上了楼梯。

“你整出来了吗?” 她把 AD 钙递给我,问道。

“没有。” 我说。

空气好像安静了一会儿。我想我迟早是要说的。那不如现在就说:“但很抱歉,我把那个文件弄坏了,现在里面的文件应该都被删除了。”

“什么?”

“我在你走的这段时间自己试了几个答案,超出了它规定的错误次数。现在里面的文件应该已经消失了。”

张诺听我这么说,连忙跑到我的电脑前。看着文件阅览器上的可执行文件大小发生了变化,没有再说什么。我想,她信了我的话。

“诺啊,你别生气。”

“呵呵。我不生气,我为什么要生气。你不是很能耐吗?我临走前是怎么说的,要你有思路联系我一声,我立刻赶回来。你为什么不听呀?” 她没有多在我旁边停留哪怕一秒钟,背过身去走到离我最远的一张桌子前坐了下来。

我没什么能说的,看来这个坏人我是当定了。既然如此,我不妨更坏一些,把话都说清楚吧。而能听明白其中多少含义,就是张诺她自己的事了。

我清理了一下嗓子,说道:“哎,张诺。你有没有想过,凯茜的选择可能与你有关呢?毕竟你是她最好的朋友呀。”

张诺和我在一起的时候很少生气,但她应该知道我的脾气。一般来说,倘若我惹到了谁,只要道理不在我身上,我肯定是会第二时间赔礼道歉的。她临走前特意叮嘱了我,并且我知道这件事也是她的原因,因此这件事在她看来我明显不占理。她应该以为我会十分诚恳地向她请求原谅吧?然而我如今避开那件事不谈,选择换一个话题,八成在她的意料之外。

“什么意思?”

“我的意思是,凯茜她选择离开,是否与你的所作所为脱不开干系呢?”

“李佳辛你什么意思?你难道说,是我?……”

“我没有那个意思,我并非暗示你是第三个问题的答案。我只是想说,你作为她最要好的朋友,在她曾经面临困难时,你是否给过她足够的帮助呢?诺啊,你这么聪明,应该能看出她实际的精神状态并不好。” 我说。

张诺没有看我,一下子靠在座位上,望着惟心馆的天花板:“你说得对。我第一次注意到还是这个学期开始时。当时……我也忘了发生了什么,总之给了我一种感觉——她不太对劲。”

“过后,隔三岔五我们就约一次饭,黄小川、渝锦汇、御晋轩,我们经常去吃这些。她跟我说压力很大,其实说真的,我不知道她的压力来自于哪里。依我看来,在她的专业拿一个前 5\% 并不算什么困难的事。所以我也不知道如何安慰她,只是说:‘放轻松一点,放松一点可能反而会有更好的成绩’ 之类的话。”

“她也是你们济心学堂的,她说你们那里很卷。我问她济心的 ‘卷’ 是什么,她说不知道,只是大家都在说这个字,大家压力都很大。记得她好像给我举了个例子,是听说佳辛你为了卷过别人,两天时间写了 23000 字总共 50 多页的报告,是真的吗?算了,不管你是不是真的,我一天能堆出来的就顶你两天多了,于是更不能理解你们的 ‘卷’ 到底是什么。”

“在那之后,我找熟悉凯茜的人了解了一下,听说她的排名还不低的,而且也是拿过一些荣誉。只是凯茜可能真的厌倦了那样的生活,所以才这么说吧。后面我跟她讲:‘你们这个专业只要专业实力够用就行了,那里需要绩点这种东西嘛’,希望这样能让她少一点压力。”
张诺应该没有骗我,因为凯茜的文件中,也是这样写的。

既然都到了这一步,我也就踏出那一步吧:“或许不是这个学期开始,而是很早以前……可能早到她在大学重新认识你的那一天。”

张诺没有像我想象中的那样反驳我,反而闭上了眼睛:“佳辛,你是知道什么的吧。”

“我不知道。”

“既然不知道为什么要下这样的推断。”

“我的直觉。”

“直觉……吗……”

“是的,直觉。”

“你是理工科生,直觉也要有个所以然。”

“你太优秀了,诺。”

“什么?”

“我说你太优秀了。”

“谢谢,但我并不清楚这件事,只是大家都这么说。”

“我说的是你太优秀了。优秀到你周围的人无法呼吸。”

“再说句谢谢,但是为什么。”

“你这句 ‘为什么’ 就是最好的论据。”

“我还是不懂。”

“人很脆弱。”

“是的,但你又在说什么。”

“我说人很脆弱。”

“我知道人很脆弱。”

“我说的是人很脆弱。脆弱到你无法想象。”

“是我不知道的程度吗。”

“应该是的,因为你很难共情。”

“什么?”

“我说你很难共情。”

“我听到了,或许是,但佳辛你不要再说了。”

“我说的是你很难共情。你很优秀,人很脆弱,你很难共情,你周围的人会因为你的优秀和强大而显得格外低劣和脆弱。你无法理解他们,你很难体会你这个个体是多么的特殊,你无法想象像凯茜那样的同学每个黑夜在面临些什么,你不知道对你来说如鱼得水的环境对他们来说意味着什么。”

“……”

“你不知道济心的 ‘卷’,那我来告诉你。她所在的专业 4.5 以上超过 50\%,上学期满绩 10 几名。可能那在你看来也没什么大不了的。但那都是人,而不是数字。诚然有人不需要排名,但也有人格外看重它。我明白,你听不懂的。”

“……”

“你们这届开始有了大类招生与专业分流。我们三年学的科目,他们要两年学完。我们的课程没有什么连续性,每一个大作业都是从零开始,你能想象去年 12 月中下旬的他们吗。”

“……”

“你的安慰是苍白的。如果你不是 1\%,而是和凯茜一样的排名,哪怕你说一样的话,对她来说都是冬季的暖阳。”

“……”

我还有很多话要说,但一时竟说不出来。我转过椅子,打算趁这个功夫修几个课设项目中的后端 bug。这周是第八周,虽然还有将近两个月的时间,但时间也不早了。大学中的组队项目向来是憋到最后憋不住的人干活,而我连第一周都憋不住。

下一秒,我忘记了张诺、忘记了林凯茜、忘记了内卷、想起了 branch、想起了 tag、想起了 commit……

“……”

“数据库怎么又连不上了?”

“……”

~\\

“呜……”

我听到不对的声音,于是立刻站起身来——

是张诺哭了。她哭了。

这是我第一次看到她哭。

我之前一直以为她只会笑。她明明每次见了我都是 “哈哈哈哈哈哈”,或许因此,一直微笑对待万难的她在我眼里永远有一种其他人不具备的美。

但现在,我发现我错了,她哭起来也很美。可却是如此令人心碎。

她都知道的。我说的事情她都知道吧。

但为什么她没有反驳我。

我动弹不得,不知怎么办才好,整个人只配杵在地上。我只觉得刚才说的过火了一点,遣词造句有点带表演性质,她肯定也能听出来,可或许对她来说终究太过分了。

或许张诺痛骂了我一顿,然后我安慰了她两个小时。

或许张诺直接跑出了惟心馆,而我追她追了大半个校区。

或许张诺一直闷着声哭,我也一直盯着地板发神。

我不知道那之后到底是怎样的,我没有记忆了,我什么都不记得了。那天我是怎么和她分开、怎么回的寝室,我都不记得了。

~\\

“话说晚上吃教师食堂吗” 可能是前天凌晨 2 点肝得太狠了,我一觉睡到大下午,醒来拿起枕头旁的手机,看到电量只剩 20\%。张诺给我发了两条消息,上面这句是第二条,第一条是:“话说中午吃教师食堂吗”。

我立刻回了她消息,翻身下床。等我到春谷苑二楼时,看到她坐在门口笑着看手机,我想:“大概没事了吧?”

我知道她很难忘记,这也不应该被忘记。

论学校内的评价体系指标,她们哪一点都比我优秀得多,但我希望都能像我一样,有时也似一坨烂泥。

\section{后记}

一周后,惟心馆周围突然多了许多工人,他们在二楼平台上增设的 LED 灯管不可计数。

我在图书馆九楼望去。

夜里,惟心馆亮极了。

\end{document}
